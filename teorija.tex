\documentclass[a4paper,oneside,12pt,fleqn]{article}
\usepackage[slovene]{babel}
\usepackage[utf8]{inputenc}
\usepackage[T1]{fontenc}
\usepackage{url}
\usepackage{graphicx}
\usepackage{epstopdf}
\usepackage[usenames]{color}
\usepackage[reqno]{amsmath}
\usepackage{amssymb}
\usepackage{icomma}
\usepackage[bookmarks, colorlinks=true, unicode=true,%
linkcolor=black, anchorcolor=black, citecolor=black, filecolor=black,%
menucolor=black, runcolor=black, urlcolor=black%
]{hyperref}
\hypersetup{pdftitle={Teorija pri matematiki}}
\hypersetup{pdfauthor={Jure Slak}}
\hypersetup{pdfsubject={Zapiski}}
\usepackage[
    paper=a4paper,
    top=2.5cm,
    bottom=2.5cm,
%    textheight=24cm,
    textwidth=15cm,
    ]{geometry}


\usepackage{amsthm}
{
\newtheorem{izrek}{Izrek} %[section]
\newtheorem{dokaz}[izrek]{Dokaz}
\newtheorem{trditev}[izrek]{Trditev}
\newtheorem{posledica}[izrek]{Posledica}
{
\theoremstyle{definition}
\newtheorem{definicija}{Definicija}
}
}

\usepackage{makeidx}
\makeindex
\usepackage{auto-pst-pdf}
\usepackage{pst-plot}
\usepackage{pst-math}
\usepackage{pst-eucl}
\usepackage{subfigure}
\usepackage{multido}
\usepackage{gensymb}
\usepackage{tabularx}
\usepackage{pbox}
\usepackage{units}
\usepackage{mathtools}

\def\R{\ensuremath{\mathbb R}}
\def\N{\ensuremath{\mathbb N}}
\def\Z{\ensuremath{\mathbb Z}}
\def\C{\ensuremath{\mathbb C}}
\def\Q{\ensuremath{\mathbb Q}}
\mathchardef\mhyphen="2D

\newcommand{\edot}{\;\;\;.}
\newcommand\krat\cdot
\newcommand{\Rel}{\mathcal{R}}
\newcommand{\comment}[1]{\ensuremath{\qquad\backslash\backslash\;}\textnormal{#1}}

\makeatletter
\newcommand{\rom}[1]{\romannumeral #1}
\newcommand{\Rom}[1]{\expandafter\@slowromancap\romannumeral #1@}
\makeatother

\makeatletter
\newcommand{\hyperanchor}[1]{\Hy@raisedlink{\hypertarget{#1}{}}}
\makeatother

\newcommand{\beforecaptionskip}{\vspace{-12pt}}
\newcommand{\arccot}{\ensuremath{\operatorname{arccot}}} % arccot
\newcommand{\tg}{\ensuremath{\operatorname{tg}}} % tg 
\newcommand{\ctg}{\ensuremath{\operatorname{ctg}}} % ctg
\newcommand{\asimptota}{\psline[linecolor=lightgrey, linestyle=dashed, linewidth=.5pt]}
\newcommand{\oznaka}{\psline[linecolor=red, linestyle=dotted]}
\newcommand{\ii}{\ensuremath{\imath}}
\newcommand\kos\cos
\def\deg{\degree}

% multi line table cell with 0.5mm margin
\newcommand{\mltc}[1]{\pbox{21cm}{\vspace{0.5mm} #1 \vspace{0.5mm}}}
% new column type that expands and centers content
\newcolumntype{C}{>{\centering\arraybackslash}X}

% your definitions here ... 

% arrow resetings
\renewcommand\implies\Rightarrow
\renewcommand\iff\Leftrightarrow

%counter settings
\numberwithin{equation}{section}

\title{Teorija pri matematiki}
\author{Jure Slak}
\date{\today}

% space settings
\setlength{\parindent}{0pt}
\setlength{\parskip}{5pt}

\newenvironment{itemize*}%
{
\vspace{-12pt}%
\begin{itemize}%
\setlength{\itemsep}{0pt}%
\setlength{\parskip}{2pt}}%
{\end{itemize}}

\newenvironment{enumerate*}%
{
\vspace{-12pt}%
\begin{enumerate}%
\setlength{\itemsep}{0pt}%
\setlength{\parskip}{2pt}}%
{\end{enumerate}}

% aligned list
\newenvironment{xlist}[1][\rule{0.75 cm}{0cm}]{%
\vspace{-14pt}
  \begin{list}{}{%
    \settowidth{\labelwidth}{#1:}
    \setlength{\labelsep}{0.5cm}
    \setlength{\leftmargin}{\labelwidth}
    \addtolength{\leftmargin}{\labelsep}
    \addtolength{\leftmargin}{20pt}
    \setlength{\rightmargin}{0pt}
    \setlength{\parsep}{0.5ex plus 0.2ex minus 0.1ex}
    \setlength{\itemsep}{0 ex plus 0.2ex}
    \renewcommand{\makelabel}[1]{##1:\hfil}
    }
  }
{\end{list}}

% fixind description and list labels (using hyperref}
\makeatletter
\let\orgdescriptionlabel\descriptionlabel
\renewcommand*{\descriptionlabel}[1]{%
\let\orglabel\label
\let\label\@gobble
\phantomsection
\edef\@currentlabel{#1}%
\let\label\orglabel
\orgdescriptionlabel{#1}%
}
\makeatother

\begin{document}
%set spacing 2
\setlength{\abovedisplayskip}{3pt}
\setlength{\belowdisplayskip}{6pt}

% \renewcommand{\listfigurename}{Kazalo slik} 

\thispagestyle{empty}
% naslovnica

\mbox{}\\
\vspace{3ex}

\noindent \scalebox{1.8}[2.1]{\textsf{{\Huge \bfseries Teorija pri}}} \\[5pt]
\scalebox{1.8}[2.1]{\textsf{{\Huge \bfseries matematiki}}} \\[60pt]
\scalebox{1}[1.2]{\textsf{\Huge \bfseries Jure Slak}}

\vspace{50pt}
\noindent \mbox{} \hspace{-20pt} \scalebox{35}{$e^{\pi i}$}
\pagebreak

\thispagestyle{empty}
\tableofcontents
\pagebreak

\section{Izjave}
\label{sec:izjave}
\textbf{Izjava} je smiseln povedni stavek, ki mu lahko določimo njegovo vrednost.

\textbf{Negacija} izjave $A$ je nova izjava, ``Ni res, da drži $A$.'' ($\neg A$), ki je pravilna, če je izjava $A$
napačna in obratno. 

\textbf{Konjunkcija} izjav $A$ in $B$ je nova izjava ``$A$ in $B$.'' ($A \land
B$), ki je pravilna le, če sta izjavi $A$ in $B$ pravilni.

\textbf{Disjunkcija} izjav $A$ in $B$ je nova izjava ``$A$ ali $B$.'' ($A \lor B$), 
ki je pravilna, ko je pravilna vsaj ena izmed izjav $A$ in $B$.

\textbf{Implikacija} izjav $A$ in $B$ je nova izjava ``Če $A$, potem sledi $B$.'' ($A
\implies B$), ki je napačna samo v primeru, da je prva izjava pravilna, druga pa napačna.

\textbf{Ekvivalenca} izjav $A$ in $B$ je nova izjava ``Če $A$, natanko takrat $B$.'' ($A 
\iff B$), ki je pravilna, če imata izjavi enako vrednost.

\section{Množice}
\label{sec:mnozice}
\textbf{Množica} je skupina elementov, ki jih druži neka skupna lastnost. 

\textbf{Prazna množica} je množica brez elementa. ($\emptyset$)

\textbf{Univerzalna množica} ($\mathcal{U}$) je množica, ki vsebuje vse elemente, ki jih preučujemo.

Množica $A$ je \textbf{podmnožica} $B$ ($A \subset B$), če je vsak element množice $A$ tudi element
množice $B$.

Dve množici sta \textbf{enaki}, če imata iste elemente.

\textbf{Unija} množic $A$ in $B$ ($A \cup B$) je nova množica, ki vsebuje elemente, ki so v množici $A$ ali v
množici $B$. ($A \cup B = \left\{x:x \in A \lor x \in B \right\}$)

\textbf{Presek} množic $A$ in $B$ ($A \cap B$) je nova množica, ki vsebuje elemente, ki so v množici $A$ in v
množici $B$. ($A \cup B = \left\{x:x \in A \land x \in B \right\}$)

\textbf{Razlika} množic $A$ in $B$ ($A - B$ ali $A \setminus B$) je nova množica, ki vsebuje vse elemente, ki so v drugi množici
v prvi pa ne. ($A - B = \left\{x \in A; x \notin B \right\}$)

\textbf{Komplement} množice $A$ ($A^c$) je nova množica, ki vsebuje vse elemente, ki niso
v množici $A$. ($A^c = \mathcal{U} - A$)

\textbf{Moč} množice je število njenih elementov. ($|A|$) 

\textbf{Potenčna množica} množice $A$ je množica vseh podmnožic množice $A$.
($|\mathcal{P}(A)| = 2^{|A|}$). 

\textbf{Kartezični produkt} množic $A$ in $B$ ($A \times B$) je nova množica, ki vsebuje urejene pare, v katerih
je prvi element iz množice $A$, drugi pa iz množice $B$. ($A \times B = \left\{ \left(a,
b \right); a \in A \land b \in B \right\}, \; \left|A \times B \right| = \left|A\right|
\krat \left|B\right|$).

\section{Preslikave}
\label{sec:preslikave}
\textbf{Preslikava}, ki množico $A$ preslika v množico $B$ ($f\!\!: A \rightarrow B,
f\!\!:a \mapsto b$), je predpis, ki vsakemu elementu iz
množice $A$ priredi natanko določen element iz množice $B$. 

Preslikava je \textbf{injektivna}, kadar se par različnih elementov iz množice $A$ preslika v par
različnih elementov iz množice $B$. ($a_1 \neq a_2 \implies f(a_1) \neq f(a_2); a_1, a_2
\in A$)

Preslikava je \textbf{surjektivna}, kadar je vsak element množice $B$ slika vsaj enega elementa iz
množice $A$. ($\forall b \in B, \exists a \in A\!: b = f(a)$)

Preslikava je \textbf{bijektivna}, če je injektivna in surjektivna hkrati.

Graf preslikave $f\!\!: A \rightarrow B$ je podmnožica kartezičnega produkta $A \times B$.

\section{Relacije}
\label{sec:relacije}
\textbf{Relacija} je odnos med elementi neke množice.
Relacija je podmnožica kartezičnega produkta.

Relacija je \textbf{refleksivna}, za vsak element v množici velja, da je element v relaciji sam s
seboj. ($\Rel \text{ refleksivna} \iff \forall a \in A\!: a \Rel a$)

Relacija je \textbf{simetrična}, kadar za vsak par elementov velja, če je prvi v relaciji z drugim,
je tudi drugi v relaciji s prvim. 
($\Rel \text{ simetrična} \iff \forall a, b \in A\!: a \Rel b \implies b \Rel a$)

Relacija je \textbf{tranzitivna}, če za vsako trojico elementov velja, če je prvi v relaciji z
drugim in drugi v relaciji s tretjim, potem je tudi prvi v relaciji s tretjim.
($\Rel \text{ tranzitivna} \iff \forall a, b, c \in A\!: (a \Rel b \land b \Rel c) \implies a \Rel c$)

Relacija je \textbf{ekvivalenčna}, če je refleksivna, simetrična in tranzitivna hkrati.

\section{Naravna števila}
\label{sec:naravna}
\[ \N = {1, 2, 3, 4, 5, 6, \ldots} \]
\textbf{Operacija} dvema elementoma priredi nov element.

\subsection{Zakoni}
\label{sec:naravna:zakoni}
Zakon o \textbf{komutativnosti} ali zakon o zamenjavi za množenje in seštevanje:
\[ a + b =  b + a \]
\[ a \krat b =  b \krat a \]

Zakon o \textbf{asociativnosti} ali zakon o združevanju za množenje in seštevanje:
\[ a + (b + c) = (a + b) + c \]
\[ a \krat (b \krat c) = (a \krat b) \krat c \]

Zakon o \textbf{distributivnosti} ali zakon o razčlenjevanju:
\[ a \krat (b + c) = a \krat b + a \krat c \]


\subsection{Številski sestavi}
Vsako število v desetiškem sistemu z osnovo 10 lahko zapišemo v kateremkoli sistemu
z osnovo $b$. \\
Poljubno število $a_na_{n-1}\!\ldots a_4a_3a_2a_1a_0$ pomeni:
\[ a_n \krat b^n + a_{n-1} \krat b^{n-1} + \cdots + a_2 \krat b^2 +a_1 \krat b + a_0 \]

$b$ osnova, $b \in \N \land b \geq 2$ \\
$a_i$ števka, $0 \leq a < b$

Vsako naravno število $a$ lahko zapišemo na en sam način v številskem sestavu z osnovo $b$

\subsection{Relacija deljivosti}
Število $a$ deli število $b$ natanko takrat, ko je število $b$ večkratnik števila $a$.
\[ a | b \iff b = k \krat a;\quad a, b, k \in \N \]

\textbf{Lastnosti:}
\begin{xlist}[Antisimetričnost]
  \item[refleksivnost]       $a|a$
  \item[antisimetričnost]  $a|b \land b|a \implies a = b$
  \item[tranzitivnost]      $a|b \land b|c \implies a|c$
  \item[neimenovana 1]      $a|b \land a|c \implies a|\left( b+c \right)$
  \item[neimenovana 2]      $a|b \land a|\left( b+c \right) \implies a|c$
\end{xlist}

\textbf{Dokaz antisimetričnosti:}
\begin{align}
  a|b \iff b &= k_1 \krat a; \quad k_1 \in \N \\
  b|a \iff a &= k_2 \krat b; \quad k_2 \in \N \label{eq:del:antis:akb} \\
  a &= k_2 \krat b \comment{izhaja iz definicije} \label{eq:del:antis:akb2} \\
  a &= k_2 \krat k_1 \krat a \comment{$b$ zamenjamo po definiciji, glej \eqref{eq:del:antis:akb}} \\
  k_1 \krat k_2 &= 1 \\
  k_1 &= 1, k_2 = 1 \label{eq:del:antis:kk} \comment{$k_1k_2$ je lahko 1 le, če velja ta
  vrstica} \\
  a &= b \comment{zazremo se v \eqref{eq:del:antis:akb2} in se spomnimo na \eqref{eq:del:antis:kk}}
\end{align}

\textbf{Dokaz tranzitivnosti:}
\begin{align}
  a|b \iff b &= k_1 \krat a; \quad k_1 \in \N \label{eq:del:tranz:bka} \\
  b|c \iff c &= k_2 \krat b; \quad k_2 \in \N \label{eq:del:tranz:ckb} \\
  c &= k_2 \krat b \\
  c &= \underbrace{k_2 \krat k_1}_{k_3} \krat a \comment{$b$ zamenjamo po \eqref{eq:del:tranz:bka}} \\
  c &= k_3 \krat a \implies a|c
\end{align}

\textbf{Dokaz neimenovane 1:}
\begin{align}
  a|b \iff b &= k_1 \krat a; \quad k_1 \in \N \label{eq:del:brez1:bka} \\
  a|c \iff c &= k_2 \krat a; \quad k_2 \in \N \label{eq:del:brez1:cka} \\
  b + c &= k_1 \krat a + k_2 \krat a \comment{zamenjamo po \eqref{eq:del:brez1:bka} in
  \eqref{eq:del:brez1:cka}} \\
  b + c &= (\underbrace{k_1 + k_2}_{k_3}) \krat a \\
  b + c &= k_3 \krat a \implies a|(b+c)
\end{align}

\textbf{Dokaz neimenovane 2:}
\begin{align}
  a|b \iff b &= k_1 \krat a; \quad k_1 \in \N \label{eq:del:brez2:bka} \\
  a|(b+c) \iff b+c &= k_2 \krat a; \quad k_2 \in \N \label{eq:del:brez2:bcka} \\
  b + c &= k_2 \krat a \comment{po definiciji \eqref{eq:del:brez2:bcka}} \\
  k_1 \krat a + c &= k_2 \krat a \comment{zamenjamo $b$ po \eqref{eq:del:brez2:bka}} \\
  c &= k_2 \krat a - k_1 \krat a \\
  c &= (\underbrace{k_2 - k_1}_{k_3}) \krat a \\
  c &= k_3 \krat a \implies a|c
\end{align}

\subsubsection{Kriteriji deljivosti}
\begin{align*}
  2|a &\iff 2|a_0 \\
  3|a &\iff 3|(a_0 + a_1 + \cdots + a_n) \\
  4|a &\iff 4|(10a_1+a_0) \\
  5|a &\iff 5|a_0 \\
  6|a &\iff 2|a \land 3|a \\
  8|a &\iff 8|(100a_2 + 10a_1+a_0) \\
  9|a &\iff 9|(a_0 + a_1 + \cdots + a_n) 
\end{align*}

\textbf{Praštevila} so števila, ki imajo natanko dva delitelja. Praštevil je neskončno mnogo.
Dokaz manjka. Števila, ki imajo več kot dva različna delitelja so \textbf{sestavljena} števila. 
\begin{izrek}
  \textnormal{\textbf{Osnovni izrek aritmetike:} Vsako število lahko zapišemo kot produkt
  samih praštevil.}
\end{izrek}

\begin{izrek}
  \textnormal{\textbf{Osnovni izrek o deljenju:} Za vsaki dve števili $a$ in $b$ obstajata
  natanko določeni števili $k$ in $o$, tako da velja $a = k \krat b+ o; \quad 0 \leq o < b$.}
\end{izrek}

\textbf{Največji skupni delitelj} števil $a$ in $b$ je največje število, ki deli obe števili
hkrati. (oznaka: $D(a,b)$) \\
\textbf{Najmanjši skupni večkratnik} dveh števil $a$ in $b$ je največje število, ki je deljivo z obema
številoma hkrati. (oznaka: $v(a,b)$) \\
Med $D(a,b)$ in $v(a,b)$ velja zveza: 
\[ D(a,b) \krat v(a,b) = a \krat b \edot \]
Števili sta si \textbf{tuji}, ko je njun največji skupni delitelj enak 1. \\
\textbf{Evklidov algoritem} je postopek s katerim dobimo $D(a,b)$. Zadnji od nič različen
ostanek je $D(a,b)$.

\section{Cela števila}
\label{sec:cela}
\[ \Z = \Z^- \cup \left\{ 0 \right\} \cup \Z^+ \]

\subsection{Zakoni}
\label{sec:cela:zakoni}
Vsi zakoni kot za naravna (razdelek \ref{sec:naravna:zakoni}). Poleg teh velja še: \\
Obstaja nevtralni element za seštevanje in je 0:
\[ a + 0 = a \]
Obstaja nevtralni element za množenje in je 1: 
\[ 1 \krat a = a \]
Vsota števila in nasprotnega števila je enaka 0 ali ``nasprotnost je vzajemna'': 
\[ a + (-a) = 0 \]

\subsection{Zakoni urejenosti}
\label{sec:cela:zakoniurejenosti}
Za vsako trojico števil $a$, $b$ in $c$ velja: \\
$\forall a, b, c \in \Z \!:$ 
\begin{enumerate*}
  \item $a < b \lor a = b \lor a > b$
  \item $a < b \land b < c \implies a < c$
  \item $a < b \implies a + c < b + c$
  \item $a < b \land c > 0 \implies ac < bc$
  \item $a < b \land c < 0 \implies ac > bc$
\end{enumerate*}

\section{Racionalna števila}
\label{sec:rac}
\[ \Q = \left\{\frac{a}{b}; a,b \in \Z, b \neq 0 \right\} \]

\subsection{Zakoni}
Vsi zakoni kot za cela števila (razdelek \ref{sec:cela:zakoni}). Poleg teh še: \\
Produkt števila in obratnega števila je enak 1 ali ``obratnost je vzajemna'': 
\[ a \krat a^{-1} = 1 \]

\textbf{Deljenje} je množenje z obratno vrednostjo. \\
Razširjanje ulomkov: ulomek lahko v števcu in v imenovalcu pomnožimo z enakim številom, pa
se vrednost ne spremeni:
\[ \frac{a}{b} = \frac{a \krat k}{b \krat k} \]

\textbf{Seštevanje} racionalnih števil: 
\[ \frac{a}{b} \pm \frac{c}{d} = \frac{ad}{bd} \pm \frac{cb}{bd} = \frac{ad \pm bc}{bd} \]

\textbf{Množenje} racionalnih števil:
\[ \frac{a}{b} \krat \frac{c}{d} = \frac{a \krat c}{b \krat d} \]

Vsak ulomek lahko zapišemo z \textbf{decimalnim številom}, ki je lahko končno ali periodično.
Ulomki, ki jih lahko razširimo tako, da imajo v imenovalcu potenco z osnovo 10, se
imenujejo \textbf{desetiški} ulomki. V razcepu imajo lahko le faktorja 5 in 2. Taki ulomki so končna 
decimalna števila.

\subsection{Urejenost racionalnih števil}
Ulomke lahko predstavimo na številski premici. Množica racionalnih števil je povsod enako
gosta. Med dvema racionalnima številoma je vedno še vsaj eno racionalno število.
\[ a < \frac{a+b}{2} < b; a, b \in \Q \] 

\section{Realna števila}
To je množica vseh \textbf{decimalnih} števil. Med množico \R{} in množico točk na premici
obstaja bijektivna preslikava.
\begin{izrek}
  \label{izrek:koren2nirac}
  $\sqrt{2} \notin \Q$
\end{izrek}

\textbf{Dokaz izreka \ref{izrek:koren2nirac}:} 
\begin{align*}
  A &:= \sqrt{2} \notin \Q \\
  \neg A &:= \sqrt{2} \in \Q  \comment{dokažimo trditev $\neg A$} \\
  \sqrt{2} &=  \frac{p}{q} \comment{$\sqrt{2} \in \Q$, torej se ga lahko zapiše kot okrajšan ulomek} \\
  2 &= \frac{p^2}{q^2} \\
  2q^2 &= p^2 \comment{kvadrat $p$ je sodo, torej je tudi $p$ sodo; $p = 2m$} \\
  2q^2 &= \left( 2m \right)^2 \\
  q^2 &= 2m^2 \comment{kvadrat $q$ je sodo, torej je tudi $q$ sodo; $q = 2n$} \\
  \left( 2n \right)^2 &=  2m^2 \\
  2n^2 &= m^2 \\
  &\vdots \comment{$p$ je sodo, $q$ je sodo, torej ulomek ni okrajšan, trditev je napačna} \\
  \neg A = 0 &\Rightarrow A = 1 \comment{$\sqrt{2}$ torej ni element \Q} \\
\end{align*}
\[  \sqrt{2} \notin \Q \]

\section{Absolutna vrednost}
\begin{equation}
  \label{eq:abs}
  \left| x \right| = 
  \begin{cases} 
    \hfill  x; & \text{če $x \geq 0$,} \\
    \hfill -x; & \text{če $x < 0$.}
  \end{cases}
\end{equation}

\textbf{Lastnosti:}
\begin{itemize*}
  \item $|x| \geq 0$
  \item $|x| = 0 \iff x = 0$
  \item Grafično predstavlja oddaljenost števila od izhodišča na številski premici.
  \item $|xy| = |x| \krat |y| \quad$ Absolutna vrednost produkta je enaka produktu
    absolutnih vrednosti.
  \item $|x+y| \leq |x| + |y| \quad$ Absolutna vrednost vsote je manjša ali enaka vsoti
    absolutnih vrednosti. (\textbf{trikotniška neenakost})
\end{itemize*}

\section{Intervali}
$\left[a,b \right] = \left\{x; a \leq x \leq b; \; x \in \R \right\}$ --- zaprt interval \\
$\left(a,b \right) = \left\{x; a < x < b; \; x \in \R \right\}$ --- odprt interval \\
$\left[a,b \right] = \left\{x; a \leq x < b; \; x \in \R \right\}$ --- polodprt interval \\
$\left(a,b \right] = \left\{x; a < x \leq b; \; x \in \R \right\}$ --- polzaprt interval \\
$(-\infty,\infty) = \R$

\section{Izrazi}
\textbf{Matematični izraz} je zapis sestavljen iz števil, spremenljivk, matematičnih funkcij in
operacij ter iz oklepajev, ki določajo vrstni red računanja. Da je tak zapis res
matematični izraz, mora biti tudi \textbf{smiseln:} Če namesto spremenljivk vstavimo konkretna
števila, mora biti možno izračunati vrednost izraza (vsaj za nekatere vrednosti
spremenljivk).
Primer:
\[ \frac{x + 1}{x} \]
Vrednost tega izraza lahko izračunamo za katero koli vrednost
spremenljivke x, razen za x = 0.

Dva matematična izraza sta \textbf{enakovredna}, če imata pri istih izbirah spremenljivk vedno
enako vrednost.
Primer: Zgornji izraz je enakovreden izrazu 
\[ 1 + \frac{1}{x} \]

Izraz \textbf{poimenujemo} glede na glavno računsko operacijo, ki v njem nastopa -- to je računska
operacija, ki jo izračunamo nazadnje.
Primeri: \\
$(x + 1)(x + 2)$ imenujemo produkt izrazov $(x + 1)$ in $(x + 2)$ \\
$5a + 3b - 2c$ imenujemo vsota izrazov $5a$, $3b$ in $-2c$ \\
$(2m + 3)^2$ imenujemo kvadrat izraza $(2m + 3)$

Izraz, v katerem nastopajo samo osnovne štiri računske operacije (seštevanje, odštevanje,
množenje in deljenje), imenujemo \textbf{aritmetični} izraz. Če v izrazu poleg tega nastopajo še
algebrske funkcije kot npr. korenjenje, je to \textbf{algebrski} izraz.

Pri preoblikovanju matematičnih izrazov pogosto uporabljamo naslednja dva postopka:
\textbf{faktorizacija} (preoblikovanje v produkt faktorjev)
\textbf{razčlenjevanje} (preoblikovanje v vsoto členov).

Formule za preoblikovanje izrazov:
\begin{align}
   n \in \N, \quad & a, b, c \in \R \\
   \label{eq:izr:kvadrat:dvoclenika} (a \pm b)^2 &= a^2 \pm 2ab + b^2 \comment{kvadrat
   dvočlenika} \\
   \label{eq:izr:kub:dvoclenika} (a \pm b)^3 &= a^3 \pm 3a^2b + 3ab^2 \pm b^3 \comment{kub
   dvočlenika} \\
   \label{eq:izr:kvadrat:troclenika} (a \pm b \pm c)^2 &= a^2 + b^2 + c^2 \pm 2ab \pm 2ac \pm 2bc \\
   \label{eq:izr:razlikakvadratov} a^2 - b^2 &= (a+b)(a-b) \comment{razlika kvadratov} \\
   \label{eq:izr:vsotakubov} a^3 \pm b^3 &= (a \pm b)(a^2 \mp ab + b^2) \comment{razlika ali
   vsota kubov} \\
   \label{eq:izr:vietovo} (x\pm a)(x\pm b) &= x^2 + (a\pm b)x \pm ab \comment{Viètovo
   previlo} \\
   \label{eq:izr:ananminusbnan} a^n - b^n &= (a - b)(a^{n-1} + a^{n-2}b + \cdots
   + ab^{n-2} + b^{n-1}) \\ 
   \label{eq:izr:ananminusbnansum} a^n - b^n &= (a - b)\sum_{i=0}^{n-1} a^ib^{n-i-1} \\
   \label{eq:izr:ananplubnan}  a^n + b^n &=  (a + b)(a^{n-1} - a^{n-2}b + \cdots
   - ab^{n-2} + b^{n-1}) \\
   \label{eq:izr:ananplusbnansum} a^n +b^n &= (a+b)\sum_{i=0}^{n-1} (-1)^ia^ib^{n-i-1}
   \comment{na lihe $n$}
\end{align}

\section{Potence}
\label{sec:pot}
\subsection{Potence z naravnim eksponentom}
\label{sec:pot:nar}
So krajši zapis za množenje več enakih faktorjev.
\begin{equation}
    a^n = \underbrace{a \krat a \krat a \krat a \krat \cdots \krat a}_n
    \label{eq:pot:defn}
\end{equation}

\subsubsection{Pravila za računanje}
\label{sec:pot:nar:prav}
Vsa pravila se dokaže tako, da se potenco zamenja po definiciji \eqref{eq:pot:defn} in
pogleda število faktorjev.
\begin{align}
  a^m \krat a^n &= a^{m+n} \label{eq:pot:anankratanam} \\
  (a^m)^n &= a^{m\krat n} \label{eq:pot:anannam} \\
  (a\krat b)^n &= a^n \krat b^n \label{eq:pot:akratbnan}
\end{align}

\subsection{Potence s celim eksponentom}
\label{sec:pot:cela}
\begin{equation}
  \label{eq:pot:defz}
  a^k =
  \begin{cases}
    \hfill a^k; & \text{če $k > 0$ (po definiciji \eqref{eq:pot:defn})} \\
    \hfill 1;   & \text{če $k = 0$} \\
    \hfill \frac{1}{a^{-k}}; & \text{če $k < 0$}
  \end{cases}
  \qquad k \in \Z
\end{equation}


\subsubsection{Pravila za računanje}
\label{sec:pot:cela:prav}
Vsa pravila kot za naravna števila (razdelek \ref{sec:pot:nar:prav}), dokažejo se s pravili za naravna, ali pa so dokazi
podobni tistim za naravna. Poleg teh veljajo še naslednja pravila:
\begin{align}
  \frac{a^n}{a^m} &= a^{n-m} \label{eq:pot:anandelanam} \\
  \frac{a^n}{b^n} &= \left( \frac{a}{b} \right)^n
\end{align}

\subsection{Potence z racionalnim eksponentom}
\label{sec:pot:rac}
V tem razdelku se uporabljajo koreni in pravila za računanje z njimi. Koreni so opisani
kasneje v razdelku \ref{sec:kor}, pravila pa v razdelku \ref{sec:kor:prav}.
\begin{equation}
  \label{eq:pot:defrac}
  a^{\frac{m}{n}} = \sqrt[n]{a^m}; \quad a \in \R^+ \cup \left\{ 0 \right\}; \; m \in \Z;
  \; n \in \Z - \left\{ 0 \right\}
\end{equation}

\subsubsection{Pravila za računanje}
\label{sec:pot:rac:prav}
\begin{align}
  a^{\frac{m}{n}} \krat a^{\frac{q}{p}} &= a^{\frac{mp+qn}{np}} \label{eq:pot:rac:prav:prodpot} \\
  \frac{a^{\frac{m}{n}}}{a^{\frac{q}{p}}} &= a^{\frac{mp-qn}{np}} \label{eq:pot:rac:prav:kvocpot} \\
  \left( a^{\frac{m}{n}} \right)^{\frac{q}{p}} &= a^{\frac{mq}{np}} \label{eq:pot:rac:prav:potpot} \\
  \left( a \krat b \right)^{\frac{m}{n}} &= a^{\frac{m}{n}} \krat b^{\frac{m}{n}}
  \label{eq:pot:rac:prav:potprod} \\
  \left( \frac{a}{b} \right)^{\frac{m}{n}} &= \frac{a^{\frac{m}{n}}}{b^{\frac{m}{n}}}
  \label{eq:pot:rac:prav:potkvoc} \\
\end{align}

Dokaz pravila \ref{eq:pot:rac:prav:prodpot}:
\[ a^{\frac{m}{n}} \krat a^{\frac{q}{p}} = \sqrt[n]{a^m} \krat \sqrt[p]{a^q} =
\sqrt[np]{a^{mp+qn}} = a^{\frac{mp+qn}{np}} \comment{pravilo za računanje s
koreni \eqref{eq:kor:prav:prodkor}} \]

Dokaz pravila \ref{eq:pot:rac:prav:kvocpot}:
\[ \frac{a^{\frac{m}{n}}}{a^{\frac{q}{p}}} = \frac{\sqrt[n]{a^m}}{\sqrt[p]{a^q}} =
\sqrt[np]{a^{mp-qn}} = a^{\frac{mp-qn}{np}} \comment{pravilo za računaje s koreni
\eqref{eq:kor:prav:kvockor}} \]

Dokaz pravila \ref{eq:pot:rac:prav:potpot}:
\[ \left( a^{\frac{m}{n}} \right)^{\frac{q}{p}} = \sqrt[p]{\left(\sqrt[n]{a^m} \right)^q}
= \sqrt[np]{a^{mq}} = a^{\frac{mq}{np}} \comment{pravilo za računanje s koreni
\eqref{eq:kor:prav:korkoranannaq}}\]

Dokaz pravila \ref{eq:pot:rac:prav:potprod}:
\[ \left( a \krat b \right)^{\frac{m}{n}} = \sqrt[n]{\left( a \krat b\right)^m} =
\sqrt[n]{a^m \krat b^m} = \sqrt[n]{a^m} \krat \sqrt[n]{b^m} = a^{\frac{m}{n}} \krat
b^{\frac{m}{n}} \comment{up. pravilo \eqref{eq:kor:prav:prodenakkor}} \]

Dokaz pravila \ref{eq:pot:rac:prav:potkvoc}:
\[ \left( \frac{a}{b} \right)^{\frac{m}{n}} = \sqrt[n]{\left( \frac{a}{b} \right)^m} =
\sqrt[n]{\frac{a^m}{b^m}} = \frac{\sqrt[n]{a^m}}{\sqrt[n]{b^m}} =
\frac{a^{\frac{m}{n}}}{b^{\frac{m}{n}}} \comment{uporabljeno pravilo \eqref{eq:kor:prav:ulenakkor}}
\]
\section{Koreni}
\label{sec:kor}
\begin{equation}
  \label{eq:kor:def}
  \sqrt[n]{a} = b \iff b^n = a; \quad a, b \in \R^+ \cup \{0\}, \; n \in \Z - \{0\}
\end{equation}
Če $a \in \R^-$: \\
če $n$ lih: $\sqrt[n]{a} = -\sqrt[n]{|a|}$ \\
če $n$ sod: ne obstaja v realnem.\\

Dogovor:
\begin{align*}
  \sqrt[2]{a} &= \sqrt{a}
\end{align*}

Osnovne izpeljave:
\begin{align}
  \sqrt[1]{a} &= a \nonumber \\
  \sqrt[n]{a^n} &= a \label{eq:kor:dog:nkorn} \\
  \left(\sqrt[n]{a}\right)^n &= a \label{eq:kor:dog:nkornzakn}
\end{align}

\subsection{Pravila za računanje}
\label{sec:kor:prav}
\begin{align}
  \sqrt[n]{a^m} &= \sqrt[p]{a^q} \iff mp = qn \label{eq:kor:prav:mpjeqn} \\
  \sqrt[n]{a^m} &= \sqrt[nx]{a^{mx}} \label{eq:kor:prav:razsirjanje} \\ 
  \sqrt[n]{a} \krat \sqrt[n]{b} &= \sqrt[n]{a \krat b} \label{eq:kor:prav:prodenakkor} \\
  \frac{\sqrt[n]{a}}{\sqrt[n]{b}} &= \sqrt[n]{\frac{a}{b}} \label{eq:kor:prav:ulenakkor} \\
  \sqrt[n]{a^m} \krat \sqrt[p]{a^q} &= \sqrt[np]{a^{mp+qn}} \label{eq:kor:prav:prodkor} \\
  \frac{\sqrt[n]{a^m}}{\sqrt[p]{a^q}} &= \sqrt[np]{a^{mp-qn}} \label{eq:kor:prav:kvockor} \\
  \sqrt[p]{\sqrt[n]{a}} &= \sqrt[np]{a} \label{eq:kor:prav:korkor} \\
  \left( \sqrt[n]{a^m} \right)^q &= \sqrt[n]{a^{mq}}  \label{eq:kor:prav:koranannaq} \\
  \sqrt[p]{\left( \sqrt[n]{a^m} \right)^q} &= \sqrt[np]{a^{mq}} \label{eq:kor:prav:korkoranannaq}
\end{align}

Pri dokazih se uporabljajo pravila za računanje s potencami s celim eksponentom (razdelek
\ref{sec:pot:cela:prav}).\\[5pt]
Dokaz pravila \ref{eq:kor:prav:mpjeqn}:
\begin{align*}
  \sqrt[n]{a^m} &= \sqrt[p]{a^q} \;/^{np} \\
  \left(\sqrt[n]{a^m}\right)^{np} &= \left(\sqrt[p]{a^q} \right)^{np} \comment{upoštevamo
  osnovno izpeljavo \eqref{eq:kor:dog:nkornzakn}} \\
  \left( a^m \right)^p &= \left( a^q \right)^n \\
  a^{mp} &= a^{qm} \\
  mp &= qn
\end{align*}

Dokaz pravila \ref{eq:kor:prav:razsirjanje}:
\begin{align*}
  \sqrt[n]{a^m} &= \sqrt[nx]{a^mx} \\
  mnx &= nmx \comment{glej pravilo \eqref{eq:kor:prav:mpjeqn}}
\end{align*}

Dokaz pravila \ref{eq:kor:prav:prodenakkor}:
\begin{align*}
  \sqrt[n]{a} \krat \sqrt[n]{b} &= x \\
  \left( \sqrt[n]{a} \krat \sqrt[n]{b} \right)^n &= x^n \\
  \left(\sqrt[n]{a}\right)^n \krat \left(\sqrt[n]{b} \right)^n &= x^n \\
  a \krat b &= x^n \\
  x &= \sqrt[n]{a\krat b} \comment{upoštevamo definicijo korena \eqref{eq:kor:def}}
\end{align*}

Dokaz pravila \ref{eq:kor:prav:ulenakkor}:
\begin{align*}
  \frac{\sqrt[n]{a}}{\sqrt[n]{b}} &= x \\
  \left( \frac{\sqrt[n]{a}}{\sqrt[n]{b}} \right)^n &= x^n \\
  \frac{\left( \sqrt[n]{a} \right)^n}{\left( \sqrt[n]{b} \right)^n} &= x^n \\
  \frac{a}{b} &= x^n \\
  x &= \sqrt[n]{\frac{a}{b}} \comment{upoštevamo definicijo korena \eqref{eq:kor:def}}
\end{align*}

Dokaz pravila \ref{eq:kor:prav:prodkor}:
\begin{align*}
  \sqrt[n]{a^m} \krat \sqrt[p]{a^q} &= x \;/^{np} \\
  \left( a^m \right)^p \krat \left( a^q \right)^n &= x^{np} \\
  a^{mp} \krat a^{qn} &= x^{np} \\
  a^{mp + qn} &= x^{np} \\
  x &= \sqrt[np]{a^{mp+qn}} \comment{upoštevamo definicijo korena \eqref{eq:kor:def}}
\end{align*}

Dokaz pravila \ref{eq:kor:prav:kvockor}:
\begin{align*}
  \frac{\sqrt[n]{a^m}}{\sqrt[p]{a^q}} &= x \; ^{np} \\
  \frac{a^{mp}}{a^{qn}} &= x^{np} \\
  a^{mp-qn} &= x^{np} \\
  x &= \sqrt[np]{a^{mp-qn}} \comment{upoštevamo definicijo korena \eqref{eq:kor:def}}
\end{align*}

Dokaz pravila \ref{eq:kor:prav:korkor}:
\begin{align*}
   \sqrt[p]{\sqrt[n]{a}} &= x \\
   \sqrt[n]{a} &= x^p \\
   a &= \left( x^p \right)^n \\
   a &= x^{np} \\
   x &= \sqrt[np]{a} \comment{upoštevamo definicijo korena \eqref{eq:kor:def}}
\end{align*}

Dokaz pravila \ref{eq:kor:prav:koranannaq}:
\begin{align*}
  \left( \sqrt[n]{a^m} \right)^q &= x \\
  \left( \sqrt[n]{a^m} \right)^{qn} &= x^n \comment{upoštevamo osnovno izpeljavo \eqref{eq:kor:dog:nkornzakn}} \\
  \left( a^m \right)^q &= x^n \\
  a^{mq} &= x^n \\
  x &= \sqrt[n]{a^{mq}}
\end{align*}

Dokaz pravila \ref{eq:kor:prav:korkoranannaq}:
\[  \sqrt[p]{\left( \sqrt[n]{a^m} \right)^q} = \sqrt[p]{\sqrt[n]{a^{mq}}} =
\sqrt[np]{a^{mq}} \comment{upoštevamo pravili \eqref{eq:kor:prav:korkor} in
\eqref{eq:kor:prav:koranannaq}.} \]

\section{Logaritmi}
\label{sec:log}
\begin{equation}
  \label{eq:log:def}
  \log_a\! x = y \iff a^y = x 
\end{equation}

Osnovne izpeljave iz definicije:
\begin{align}
  \log_a\! a &= 1 \label{eq:log:logaaje1} \\
  \log_a\! 1 &= 0 \label{eq:log:log1je0} \\
  a^{\log_a\! x} &= x \label{eq:log:analogax} \\
  \log_a\! a^y &= y \label{eq:log:logaanay}
\end{align}

Dogovora: \\
\[ \log_{10}\!x = \log x \]
\[ \log_{e}\!x = \ln x \] 

\vspace{-24pt}

\hspace{95pt}
\footnote{Matematiki za logaritem z osnovo $e$ pogosto uporabljajo
kar zapis $\log x$.}

\subsection{Pravila za računanje}
\label{sec:eq:log:prav}
Logaritem potence je enak produktu med eksponentom in logaritmom osnove:
\begin{equation}
  \log_a\! x^n = n\log_a\! x \label{eq:log:prav:loganan}
\end{equation}
Logaritem produkta je enak vsoti logaritmov posameznih faktorjev:
\begin{equation}
  \log_a\! xy = log_a\!x+ \log_a\!y \label{eq:log:prav:logprod}
\end{equation}
Logaritem kvocienta je enak razliki med logaritmom števca in imenovalca:
\begin{equation}
  \log_a\!\frac{x}{y} = \log_a\!x-\log_a\!y \label{eq:log:prav:logkvoc}
\end{equation}

Dokaz pravila \ref{eq:log:prav:loganan} (upoštevamo osnovni izpeljavi \eqref{eq:log:analogax} in
\eqref{eq:log:logaanay}):
\[ \log_a\!x^n = \log_a\!\left( a^{\log_a\!x} \right)^n = \log_a\!\left( a^{n\log_a\!x}
\right) = n \krat \log_a\!x  \]

Dokaz pravila \ref{eq:log:prav:logprod} (upoštevamo osnovni izpeljavi \eqref{eq:log:analogax} in
\eqref{eq:log:logaanay}):
\[ \log_a\!xy = \log_a\!\left( a^{\log_a\!x} \krat a^{\log_a\!y} \right) = \log_a\!\left(
a^{\log_a\!x+\log_a\!y} \right) = \log_a\!x + \log_a\!y \]

Dokaz pravila \ref{eq:log:prav:logkvoc} (upoštevamo pravili \eqref{eq:log:prav:logprod}  in
\eqref{eq:log:prav:loganan}):
\[ \log_a\!\frac{x}{y} = \log_a\!xy^{-1} = \log_a\!x + \log_a\!y^{-1} = \log_a\!x -
\log_a\!y \]

Prehod na novo osnovo:
\begin{align*}
  \log_b\!a^y &= \log_b\!x \comment{po definiciji \eqref{eq:log:def} je $a^y$ enak $x$} \\
  y \krat \log_b\!a &= \log_b\!x \comment{uporabimo pravilo \eqref{eq:log:prav:loganan}} \\
  y &= \frac{\log_b\!x}{\log_b\!a} \\
  \log_a\!x &= \frac{\log_b\!x}{\log_b\!a} \comment{po definiciji je $y$ enak $\log_a\!x$}
\end{align*}
Iz tega izpeljemo zvezo:
\[ \log_a\!x = \frac{1}{\log_x\!a} \]

\section{Koordinatni sistem}
\label{sec:koor}
\subsection{Pravokotni, v ravnini}
\label{sec:koor:pravrav}
Dve pravokotni osi. \\
$x$ --- abscisna os \\
$y$ --- ordinatna os
\[ \mathcal{M} = \left\{ (x,y); \; x, y \in \R \right\} = \R \times \R = \R^2 \]

Osi razdelita ravnino na štiri \textbf{kvadrante}: \\[3pt]
\begin{tabular}{ll}
  \Rom{1} kvadrant: & $x > 0 \land y > 0$ \\
  \Rom{2} kvadrant: & $x < 0 \land y > 0$ \\
  \Rom{3} kvadrant: & $x < 0 \land y < 0$ \\
  \Rom{4} kvadrant: & $x > 0 \land y < 0$ \\
\end{tabular}
\\
Pomembni \textbf{premici}:
\begin{align*}
  y &= x  \comment{simetrala lihih kvadrantov} \\
  y &= -x \comment{simetrala sodih kvadrantov}
\end{align*}

Pas: $a < x < b$ 

\textbf{Razdalja} med dvema točkama (dokaz: Pitagorov izrek):
\begin{align*}
  A(x_1, y_1) \\
  B(x_2, y_2) \\
  d(A,B) &= \sqrt{\left( x_2 - x_1 \right)^2 + \left( y_2 - y_1 \right)^2}
\end{align*}

\textbf{Središče} daljice:
\[ S_{AB} = \left( \frac{x_a+x_b}{2}, \frac{y_a+y_b}{2} \right) \]

\textbf{Težišče} in \textbf{ploščina} trikotnika\footnote{Za razrešitev
determinante matrike glej enačbo \eqref{determinanta}. In ja, |\ \ | ne pomeni absolutne
vrednosti.}:
\[ T_{ABC} = \left( \frac{x_a+x_b+x_c}{3},\frac{y_a+y_b+y_c}{3} \right) \]
\[ p\sigma = \frac{1}{2} \begin{vmatrix}
    x_2 - x_1 & y_2 - y_1 \\
    x_3 - x_1 & y_3 - y_1 
\end{vmatrix} \comment{$\sigma$ je orientacija trikotnika.} \]
\[ p\sigma = \frac{1}{2} \left[ (x_2-x_1)(y_3-y_1) + (y_2-y_1)(x_3-x_1) \right] \]
\[ p\sigma = \frac{1}{2} \left[ x_1(y_2-y_3 + x_2(y_3-y_1) + x_3(y_1-y_2) \right] \]

Determinanta:
\begin{equation}
  \begin{vmatrix}
    a & b \\
    c & d
  \end{vmatrix} = a \krat d - b \krat c
  \label{determinanta}
\end{equation}

\subsection{Pravokotni, v prostoru}
\label{sec:koor:pravpro}
Dve pravokotni osi. \\
$x$ --- abscisna os \\
$y$ --- ordinatna os \\
$z$ --- aplikatna os
\[ \mathcal{M} = \left\{ (x,y,z); \; x, y, z \in \R \right\} = \R \times \R \times \R = \R^3 \]

Formule so enake kot v ravnini (razdelek \ref{sec:koor:pravrav}), le da vsebujejo še
tretjo koordinato.

\section{Polarni, v ravnini}
\label{sec:koor:pol}
Nekoč drugič.

\section{Funkcije}
\label{sec:fun}
\[ f(x)\!: A \rightarrow B \]
\textbf{Funkcija}, ki množico $A$ preslika v množico $B$ je predpis, ki vsakemu elementu iz množice
$A$ priredi natanko določen element iz množice $B$.

\[ f(x)\!: A \rightarrow B; \; A, B \subseteq \R \]
Funkcija je \textbf{realna}, če podmnožico realnih števil preslika v podmnožico realnih števil.

\textbf{Definicijsko območje} ($D_f$) funkcije $f$ je množica realnih števil, za katera lahko
predpis izračunamo.

\textbf{Zaloga vrednosti} ($Z_f$) funkcije $f$ je množica realnih števil, ki jih funkcija lahko
zavzame.

\textbf{Graf} ($G_f$) funkcije $f$ je množica urejenih parov $(x, y)$, pri katerih je $x$ element
definicijskega območja, $y$ pa vrednost funkcije pri $x$.
\[ G_f = \left\{ (x,y); \; x \in D_f, \; y = f(x) \right\} \]

$a$ je \textbf{ničla} funkcije, če je vrednost funkcije pri $a$ enaka 0.
\[ a \textnormal{ ničla } \iff f(a) = 0 \]

\textbf{Začetna vrednost} funkcije je vrednost funkcije pri 0.
\[ \textnormal{začetna vrednost} = f(0) \]

Funkcija je \textbf{padajoča}, če pri vsakem večjem $x$ zavzame manjšo vrednost.
\[ f(x) \textnormal{ padajoča } \iff \forall x_1, x_2 \in D_f: x_1 < x_2 \implies f(x_1) >
f(x_2) \]

Funkcija je \textbf{naraščajoča}, če pri vsakem večjem $x$ zavzame večjo vrednost.
\[ f(x) \textnormal{ naraščajoča } \iff \forall x_1, x_2 \in D_f: x_1 < x_2 \implies f(x_1) <
f(x_2) \]

Funkcija je \textbf{navzgor omejena}, ko so vse funkcijske vrednosti manjše ali enake od nekega
realnega števila $M$ (zgornja meja).
\[ f(x) \textnormal{ navzgor omejena } \iff \exists M \in \R: f(x) \leq M; \forall x \in D_f \]


Funkcija je \textbf{navzdol omejena}, ko so vse funkcijske vrednosti večje ali enake od nekega
realnega števila $m$ (spodnja meja).
\[ f(x) \textnormal{ navzdol omejena } \iff \exists m \in \R: f(x) \geq m; \forall x \in D_f \]

Funkcija je \textbf{omejena}, če je omejena navzgor in navzdol.

\textbf{Pol} je realno število, za katerega funkcija ni definirana. 

\textbf{Asimptota} je črta, ki se ji graf približuje.

Funkcija je na nekem območju \textbf{konveksna}, če za vsaki dve točki na grafu funkcije
velja, da leži graf pod daljico, ki jo določata ti dve točki.

Funkcija je na nekem območju \textbf{konkavna}, če za vsaki dve točki na grafu funkcije velja, da
leži graf nad daljico, ki jo določata ti dve točki.

Funkcija je \textbf{soda}, če za vsak $x$ iz definicijskega območja velja: $f(-x) = f(x)$.
\[ f(x) \textnormal{ soda } \iff f(-x) = f(x); \; \forall x \in D_f \]

Funkcija je \textbf{liha}, če za vsak $x$ iz definicijskega območja velja: $f(-x) = -f(x)$.
\[ f(x) \textnormal{ liha } \iff f(-x) = -f(x); \; \forall x \in D_f \]

Funkcija je na nekem območju \textbf{pozitivna}, če so vse funkcijske vrednosti na tem območju
večje od 0.

Funkcija je na nekem območju \textbf{negativna}, če so vse funkcijske vrednosti na tem območju
manjše od 0.

Funkcija je \textbf{periodična} natanko takrat, ko obstaja tak $\omega \in \R^+$, da za vsak $x$ iz
definicijskega območja velja $f(x) = f(x+\omega)$.
\[ f(x) \textnormal{ periodična } \iff \exists \omega \in \R^+: f(x) = f(x+\omega), \; \forall x \in
D_f \comment{$\omega$ --- perioda} \]

\textbf{Val} periodične funkcije je del funkcije na intervalu $[x, x+\omega], x \in D_f$.

\subsection{Premik funkcije}
\label{sec:fun:prem}
Funkcijo $y = f(x)$ premaknemo za vektor $\vec{v} = (p,q)$ (glej sliko~\ref{fig:fun:prem}).
\begin{align*}
  x &= x'-p \\
  y &= y'-q \\
  y &= f(x) \\
  y' - q &= f(x'-p) \\
  y' &= f(x'-p) + q
\end{align*}
Parameter $p$ vpliva na \textbf{premik}
po $x$ osi (levo -- desno), parameter $q$ pa na premik po $y$ osi (gor -- dol).

\begin{figure}[ht]
  \begin{center}
      \begin{pspicture*}(-3.5,-3.5)(3.5,3.5)
        \psaxes[labels=none]{->}(0,0)(-3.5,-3.5)(3.5,3.5)
        \psplot[plotstyle=curve,linecolor=black, linewidth=.5pt]{-3.5}{3.5}{x -1 mul 3 exp}
        \psplot[plotstyle=curve,linecolor=black, linewidth=.5pt]{-3.5}{3.5}{x 2 sub -1 mul 3 exp 1 add}
        \psline[linecolor=black, linewidth=1pt]{->}(-1,1)(1,2)
        \oznaka(-1,1)(0,1)
        \oznaka(-1,1)(-1,0)
        \oznaka(1,2)(0,2)
        \oznaka(1,2)(1,0)
        \uput[0](0,1){$y$}
        \uput[180](0,2){$y'$}
        \uput[-90](-1,0){$x$}
        \uput[-90](1,0){$x'$}
        \uput[180](-1,1){$T(x,y)$}
        \uput[60](-1,1){$\vec{v}$}
        \uput[45](1,2){$T'(x',y')$}
      \end{pspicture*}
  \end{center}
  \beforecaptionskip
  \caption{Premik funkcije.}
  \label{fig:fun:prem}
\end{figure}

\subsection{Razteg funkcije}
\label{sec:fun:razt}
Funkcijo $y = f(x)$ raztegnemo s parametroma $a$ in $b$.
\[ y = a \krat f\left( \frac{x}{b} \right) \]
Parameter $a$ predstavlja razteg v smeri $y$ osi, če je negativen, se graf preslika čez
$y$ os. Odvisnost funkcije od parametra $a$ je prikazana na sliki~\ref{fig:fun:a}. 

Parameter $b$ predstavlja razteg v smeri $x$ osi, če je negativen, se graf
preslika čez $x$ os. Odvisnost funkcije od parametra $b$ je prikazana na sliki~\ref{fig:fun:b}. 

\begin{figure}[ht]
  \begin{center}
    \subfigure[Parameter $a$.]{
      \label{fig:fun:a}
      \begin{pspicture*}(-3.5,-3.5)(3.5,3.5)
        \psaxes[labels=none]{->}(0,0)(-3.5,-3.5)(3.5,3.5)
        \psplot[plotstyle=curve,linecolor=black, linewidth=1pt]{-3.5}{3.5}{x 2 exp}
        \psplot[plotstyle=curve,linecolor=blue, linewidth=1pt]{-3.5}{3.5}{x 2 exp 4 mul}
        \psplot[plotstyle=curve,linecolor=red, linewidth=1pt]{-3.5}{3.5}{x 2 exp 0.3 mul}
        \psplot[plotstyle=curve,linecolor=green, linewidth=1pt]{-3.5}{3.5}{x 2 exp -1 mul}
        \uput[0](1.8,3){1}
        \uput[0](0.8,3){4}
        \uput[0](2,1){0.3}
        \uput[0](1.8,-3){-1}
      \end{pspicture*}
    }
    \subfigure[Parameter $b$.]{
      \label{fig:fun:b}
      \begin{pspicture*}(-3.5,-3.5)(3.5,3.5)
        \psaxes[labels=none]{->}(0,0)(-3.5,-3.5)(3.5,3.5)
        \psplot[plotstyle=curve,linecolor=black, linewidth=1pt]{-3.5}{3.5}{x 3 exp}
        \psplot[plotstyle=curve,linecolor=blue, linewidth=1pt]{-3.5}{3.5}{x 2.5 div 3 exp}
        \psplot[plotstyle=curve,linecolor=red, linewidth=1pt]{-3.5}{3.5}{x 0.3 div 3 exp}
        \psplot[plotstyle=curve,linecolor=green, linewidth=1pt]{-3.5}{3.5}{x -1 div 3 exp}
        \uput[0](1.5,3){$1$}
        \uput[0](0.4,3){$0.3$}
        \uput[0](2.5,1){$2.5$}
        \uput[0](-1.5,3){$-1$}
      \end{pspicture*}
    }
  \end{center}
  \beforecaptionskip
  \caption{Odvisnost funkcije od parametrov $a$ in $b$.}
  \label{fig:fun:ab}
\end{figure}


\subsection{Inverzna funkcija}
\label{sec:fun:inv}
\textbf{Inverzna funkcija} funkcije $f(x)$ je funkcija $f^{-1}(x)$, ki jo dobimo tako, da v prvotni
funkciji zamenjamo vlogo odvisne in neodvisne spremenljivke, ter izrazimo novo neodvisno
spremenljivko. \textbf{Grafično} dobimo graf $f^{-1}(x)$ tako, da graf prvotne funkcije preslikamo
čez simetralo lihih kvadrantov. Inverzno funkcijo lahko dobimo samo na območjih, ko je
prvotna funkcija \textbf{injektivna}.\footnote{Za definicijo injektivnosti glej razdelek
\ref{sec:preslikave}.}

\subsection{Linearna funkcija}
\textbf{Linearna funkcija} je vsaka funkcija oblike $y = kx + n$.
\label{sec:fun:lin}
\begin{align*}
  y = kx + n \comment{explicitna} \\
  ax + by + c = 0 \comment{implicitna} \\
  \frac{x}{m} + \frac{y}{n} = 1 \comment{odsekovna}
\end{align*}

$k$ --- smerni koeficient (vzporedne  premice imajo enak smerni koeficient) \\
$n$ --- začetna vrednost, odsek na ordinatni osi \\
$m$ --- odsek na abscisni osi

\textbf{Snop} premic na sliki~\ref{fig:fun:lin:snop}, \textbf{šop} premic na sliki~\ref{fig:fun:lin:sop}.

\begin{figure}[ht]
  \begin{center}
    \subfigure[Snop premic.]{
      \label{fig:fun:lin:snop}
      \begin{pspicture*}(-2.0,-2.0)(2.0,2.0)
        \psaxes[labels=none]{->}(0,0)(-2.0,-2.0)(2.0,2.0)
        \psplot[plotstyle=curve,linecolor=black, linewidth=1pt]{-2.0}{2.0}{x}
        \psplot[plotstyle=curve,linecolor=black, linewidth=1pt]{-2.0}{2.0}{x 1.5 sub}
        \psplot[plotstyle=curve,linecolor=black, linewidth=1pt]{-2.0}{2.0}{x 0.5 add}
        \psplot[plotstyle=curve,linecolor=black, linewidth=1pt]{-2.0}{2.0}{x 1.5 add}
      \end{pspicture*}
    }
    \subfigure[Šop premic.]{
      \label{fig:fun:lin:sop}
      \begin{pspicture*}(-2.0,-2.0)(2.0,2.0)
        \psaxes[labels=none]{->}(0,0)(-2.0,-2.0)(2.0,2.0)
        \psplot[plotstyle=curve,linecolor=black, linewidth=1pt]{-2.0}{2.0}{x 0.5 add}
        \psplot[plotstyle=curve,linecolor=black, linewidth=1pt]{-2.5}{2.5}{x 0.5 add 0.5 mul}
        \psplot[plotstyle=curve,linecolor=black, linewidth=1pt]{-2.5}{2.5}{x 0.5 add -2 mul}
        \psplot[plotstyle=curve,linecolor=black, linewidth=1pt]{-2.5}{2.5}{x 0.5 add -0.5 mul}
      \end{pspicture*}
    }
  \end{center}
  \beforecaptionskip
  \caption{Posebni medsebojni legi premic.}
  \label{fig:fun:lin:snopsop}
\end{figure}

\subsection{Potenčna funkcija}
\label{sec:fun:pot}
\textbf{Potenčna funkcija} je vsaka funkcija oblike: $f(x) = x^n; \; n \in \Z  - \left\{ 0,1 \right\}$.
Poznamo štiri glavne \textbf{grafe} potenčne funkcije, ki se delijo glede na eksponent:
\begin{itemize*}
  \item pozitiven sod eksponent (slika~\ref{fig:fun:pot:sodplus})
  \item pozitiven lih eksponent (slika~\ref{fig:fun:pot:lihplus})
  \item negativen sod eksponent (slika~\ref{fig:fun:pot:sodminus})
  \item negativen lih eksponent (slika~\ref{fig:fun:pot:lihminus}).
\end{itemize*}

\begin{figure}[ht]
  \begin{center}
    \subfigure[Pozitiven sod eksponent]{
      \label{fig:fun:pot:sodplus}
      \begin{pspicture*}(-2.2,-2.2)(2.2,2.2)
        \psaxes[labels=none]{->}(0,0)(-2.2,-2.2)(2.2,2.2)
        \psplot[plotstyle=curve,linecolor=black, linewidth=1pt]{-2.2}{2.2}{x 2 exp }
      \end{pspicture*}
    }
    \subfigure[Pozitiven lih eksponent]{
      \label{fig:fun:pot:lihplus}
      \begin{pspicture*}(-2.2,-2.2)(2.2,2.2)
        \psaxes[labels=none]{->}(0,0)(-2.2,-2.2)(2.2,2.2)
        \psplot[plotstyle=curve,linecolor=black, linewidth=1pt]{-2.2}{2.2}{x 3 exp }
      \end{pspicture*}
    } \\
    \subfigure[Negativen sod eksponent]{
      \label{fig:fun:pot:sodminus}
      \begin{pspicture*}(-2.2,-2.2)(2.2,2.2)
        \psaxes[labels=none]{->}(0,0)(-2.2,-2.2)(2.2,2.2)
        \psplot[plotstyle=curve,linecolor=black, linewidth=1pt]{-2.2}{2.2}{x -2 exp }
        \asimptota(-3,.05)(3,.05)
        \asimptota(.05,-3)(.05,3)
      \end{pspicture*}
    }
    \subfigure[Negativen lih eksponent]{
      \label{fig:fun:pot:lihminus}
      \begin{pspicture*}(-2.2,-2.2)(2.2,2.2)
        \psaxes[labels=none]{->}(0,0)(-2.2,-2.2)(2.2,2.2)
        \psplot[plotstyle=curve,linecolor=black, linewidth=1pt]{-2.2}{-0.001}{x -1 exp }
        \psplot[plotstyle=curve,linecolor=black, linewidth=1pt]{2.2}{0.001}{x -1 exp }
        \asimptota(-3,.05)(3,.05)
        \asimptota(.05,-3)(.05,3)
      \end{pspicture*}
    }
  \end{center}
  \beforecaptionskip
  \caption{Grafi potenčne funkcije.}
  \label{fig:fun:pot:grafi}
\end{figure}

\subsection{Korenska funkcija}
\textbf{Korenska funkcija} je vsaka funkcija oblike $f(x) = \sqrt[n]{x}$. \\
$n$ sod: $D_f = \R^+ \cup \left\{ 0 \right\}, \; Z_f = \R^+ \cup \left\{ 0 \right\}$ \\
$n$ lih: $D_f = \R, \; Z_f = \R$ \\
\textbf{Graf} korenske funkcije je na sliki~\ref{fig:fun:kor}.

\begin{figure}[h!t]
  \begin{center}
    \begin{pspicture*}(-3.5,-3.5)(3.5,3.5)
      \psaxes[labels=none]{->}(0,0)(-3.5,-3.5)(3.5,3.5)
      \psplot[plotstyle=curve,linecolor=black, linewidth=1pt]{0}{3.5}{x sqrt}
      \psplot[plotstyle=curve,linecolor=blue, linewidth=1pt, swapaxes=true]{-3.5}{3.5}{x 3 exp}
      \uput[-10](2,2){$\sqrt{x}$}
      \uput[30](-2,-1){$\sqrt[3]{x}$}
      \oznaka(0,1)(1,1)
      \oznaka(1,0)(1,1)
      \uput[180](0,1){$1$}
      \uput[-90](1,0){$1$}
    \end{pspicture*}
  \end{center}
  \beforecaptionskip
  \caption{Graf korenske funkcije.}
  \label{fig:fun:kor}
\end{figure}

\subsection{Kvadratna funkcija}
\label{sec:fun:kvad}
Kvadratna funkcija je vsaka funkcija oblike $f(x) = ax^2 + bx + c; \; a, b, c \in \R; \; a
\neq 0$.

\textbf{Splošna oblika} kvadratne funkcije:
\begin{equation}
  f(x) = ax^2 + bx + c
  \label{eq:fun:kvad:splos}
\end{equation}

\textbf{Temenska oblika} kvadratne funkcije, teme $T(p,q)$:
\begin{equation}
  f(x) = a(x-p)^2 + q
  \label{eq:fun:kvad:tem}
\end{equation}

\textbf{Oblika za ničle} (razcep tročlenika):
\[ f(x) = a(x-x_1)(x-x_2) \]

Prehod iz splošne v temensko obliko, formule za $p$ in $q$:
\begin{align}
  f(x) &= ax^2 + bx + c \nonumber \\
  f(x) &= a(x^2 + \frac{b}{a}x) + c \nonumber \\
  f(x) &= a\left( \left( x + \frac{b}{2a}\right)^2 - \left( \frac{b}{2a} \right)^2\right) + c \nonumber \\
  f(x) &= a\left( x + \frac{b}{2a}\right)^2 - \frac{b^2}{4a} + c \nonumber \\
  f(x) &= a\left( x + \frac{b}{2a}\right)^2 - \frac{b^2-4ac}{4a} \comment{primerjamo s
  temesko obliko} \nonumber \\
  p &= -\frac{b}{2a} \label{eq:fun:kvad:p} \\
  q &= -\frac{b^2 - 4ac}{4a} = -\frac{D}{4a} \label{eq:fun:kvad:q} \\
  D &= b^2 - 4ac \comment{diskriminanta} \label{eq:fun:kvad:D}
\end{align}

\textbf{Graf} kvadratne funkcije je premaknjena in raztegnjena parabola $f(x) = x^2$. Vsako
kvadratno funkcijo v splošni obliki lahko zapišemo tudi v temenski obliki.

\subsubsection{Ničle kvadratne funkcije}
\label{sec:fun:kvad:nic}
\textbf{Ničle} kvadratne funkcije se izračunajo po formuli:
\begin{equation}
  x_{1,2} = \frac{-b \pm \sqrt{D}}{2a} \comment{$D$ zamenjamo po  definiciji
  \eqref{eq:fun:kvad:D}}
  \label{eq:fun:kvad:nicleD}
\end{equation}
\begin{equation}
  x_{1,2} = \frac{-b \pm \sqrt{b^2 - 4ac}}{2a}
  \label{eq:fun:kvad:nicle}
\end{equation}

Kvadratne funkcija ima dve različni realni ničli če $D > 0$, eno dvojno realno ničlo, če
$D = 0$ in nobene realne ničle, če je $D < 0$.
\begin{align*}
  D > 0 &\implies x_1 \neq x_2; \; x_1, x_2 \in \R \\
  D = 0 &\implies x_1 = x_2; \; x_1, x_2 \in \R \\
  D < 0 &\implies x_1 \neq x_2; \; x_1, x_2 \notin \R
\end{align*}

\textbf{Izpeljava} formule za ničle kvadratne funkcije:
\begin{align*}
  0 &= f(x) \\
  0 &= a(x-p)^2 + q \comment{temenska oblika kvadratne funkcije \eqref{eq:fun:kvad:tem}} \\
  a(x-p)^2 &= -q \\
  (x-p)^2 &= -\frac{q}{a} \\
  x-p &= \pm \sqrt{-\frac{q}{a}} \\
  x &= p \pm \sqrt{-\frac{q}{a}} \comment{zamenjamo $p$ in $q$ po 
  \eqref{eq:fun:kvad:p} in \eqref{eq:fun:kvad:q}} \\
  x &= - \frac{b}{2a} \pm \sqrt{-\frac{-\frac{D}{4a}}{a}} \\
  x &= - \frac{b}{2a} \pm \sqrt{\frac{D}{4a^2}} \\
  x &= - \frac{b}{2a} \pm \frac{\sqrt{D}}{2a} \\
  x_{1,2} &= \frac{-b \pm \sqrt{D}}{2a}
\end{align*}

\textbf{Abscisa temena} izražena z ničlami:
\[ p = \frac{x_1+x_2}{2} \]

\subsubsection{Vpliv diskriminante in parametra $a$ na parabolo}
Prikazan je na sliki~\ref{fig:fun:kvad:aD}.
\begin{figure}[ht]
  \begin{center}
    \subfigure[$a > 0$]{
      \label{fig:fun:kvad:aplus}
      \begin{pspicture*}(-7,-2)(5,2.8)
        \qline(-7,.5)(7,.5)
        \parabola(-7,3)(-5.5,0)
        \parabola(-3,3)(-1.5,0.5)
        \parabola(1,3)(2.5,1)
        \uput[0](-6.2,2.5){$D > 0$}
        \uput[0](-2.3,2.5){$D = 0$}
        \uput[0](1.8,2.5){$D < 0$}
      \end{pspicture*}
    }
    \subfigure[$a < 0$]{
      \label{fig:fun:kvad:aminus}
      \begin{pspicture*}(-7,-2)(5,2)
        \qline(-7,.5)(7,.5)
        \parabola(-7,-3)(-5.5,0)
        \parabola(-3,-3)(-1.5,0.5)
        \parabola(1,-3)(2.5,1)
        \uput[0](-6.2,-1.5){$D < 0$}
        \uput[0](-2.3,-1.5){$D = 0$}
        \uput[0](1.8,-1.5){$D > 0$}
      \end{pspicture*}
    }
  \end{center}
  \beforecaptionskip
  \caption{Vpliv diskriminante in parametra $a$ na parabolo}
  \label{fig:fun:kvad:aD}
\end{figure}

\subsubsection{Lega premice in parabole}
\label{sec:fun:kvad:legapinp}
$ax^2 +bx + c = kx + n$
\begin{itemize*}
  \item $D > 0$ --- sekanta (slika~\ref{fig:fun:kvad:pp:sek})
  \item $D = 0$ --- tangenta (slika~\ref{fig:fun:kvad:pp:tan})
  \item $D < 0$ --- mimobežnica (slika~\ref{fig:fun:kvad:pp:mim})
\end{itemize*}
Možne lege so prikazane na sliki~\ref{fig:fun:kvad:legapinp}.

\begin{figure}[ht]
  \begin{center}
    \subfigure[Sekanta.]{
      \label{fig:fun:kvad:pp:sek}
      \begin{pspicture*}(-2,-3)(2,3)
        \parabola(-2,6)(0,.5)
        \qline(-1.5,0)(2,2)
      \end{pspicture*}
    }
    \subfigure[Tangenta.]{
      \label{fig:fun:kvad:pp:tan}
      \begin{pspicture*}(-2,-3)(2,3)
        \parabola(-2,6)(0,.5)
        \qline(-.8,0)(2,1.53)
      \end{pspicture*}
    }
    \subfigure[Mimobežnica.]{
      \label{fig:fun:kvad:pp:mim}
      \begin{pspicture*}(-2,-3)(2,3)
        \parabola(-2,6)(0,1)
        \qline(-1,0)(2,1)
      \end{pspicture*}
    }
  \end{center}
  \beforecaptionskip
  \caption{Možne lege premice in parabole}
  \label{fig:fun:kvad:legapinp}
\end{figure}

\subsection{Eksponentna funkcija}
\label{sec:fun:eks}
\textbf{Eksponentna funkcija} je vsaka funkcija oblike $f(x) = a^x, \; a \in \R^+ - \left\{ 1
\right\}$.

\begin{tabular}{ll}
  a > 1 & a < 1 \\
  $D_f = \R, Z_f = \R^+$, & $D_f = \R, Z_f = \R^+$, \\
  naraščajoča, konveksna, pozitivna, & padajoča, konveksna, pozitivna, \\
  navzdol omejena & navzdol omejena \\
  graf na sliki~\ref{fig:fun:eks:avec} & graf na sliki~\ref{fig:fun:eks:amanj}
\end{tabular}

\begin{figure}[ht]
  \begin{center}
    \psset{unit=.5cm}
    \subfigure[a > 1]{
      \label{fig:fun:eks:avec}
      \begin{pspicture*}(-4.9,-4.9)(4.9,4.9)
        \psaxes[labels=none]{->}(0,0)(-4.9,-4.9)(4.9,4.9)
        \psplot[plotstyle=curve,linecolor=red, linewidth=1pt]{-4.9}{4.9}{2.7182 x exp}
        \psplot[plotstyle=curve,linecolor=blue, linewidth=1pt]{-4.9}{4.9}{1.5 x exp}
        \psplot[plotstyle=curve,linecolor=green, linewidth=1pt]{-4.9}{4.9}{10 x exp}
        \uput[180](0,4.5){$10^x$}
        \uput[0](1.5,4.5){$e^x$}
        \uput[0](2.5,2.8){$1,5^x$}
      \end{pspicture*}
    }
    \subfigure[a < 1]{
      \label{fig:fun:eks:amanj}
      \begin{pspicture*}(-4.9,-4.9)(4.9,4.9)
        \psaxes[labels=none]{->}(0,0)(-4.9,-4.9)(4.9,4.9)
        \psplot[plotstyle=curve,linecolor=red, linewidth=1pt]{-4.9}{4.9}{0.5 x exp}
        \psplot[plotstyle=curve,linecolor=blue, linewidth=1pt]{-4.9}{4.9}{0.66666667 x exp}
        \psplot[plotstyle=curve,linecolor=green, linewidth=1pt]{-4.9}{4.9}{0.1 x exp}
        \uput[180](-0.5,4.5){$\frac{1}{10^x}$}
        \uput[180](-2.2,4.5){$\frac{1}{2^x}$}
        \uput[180](-2.5,2.6){$\left( \frac{2}{3} \right)^x$}
      \end{pspicture*}
    }
    \psset{unit=1cm}
  \end{center}
  \beforecaptionskip
  \caption{Graf eksponentne funkcije.}
  \label{fig:fun:eks:graf}
\end{figure}

Vodoravna \textbf{asimptota} je $x$ os. Vse eksponentne funkcije gredo skozi točko $N(0,1)$, kar
izhaja iz definicije~\eqref{eq:pot:defz}. Vse z osnovo iz enake skupine se razlikujejo le
po \textbf{strmini} padanja in naraščanja. Lahko jih premikamo ali raztegujemo.
\[ f(x) = b\krat a^{x-p} + q \]

\subsection{Logaritemska funkcija}
\textbf{Logaritemska funkcija} je vsaka funkcija oblike $y = \log_a\!x, \; a \in \R^+ - \left\{ 1
\right\}$.
Je \textbf{inverzna} funkcija eksponentni funkciji.

\begin{tabular}{ll}
  a > 1 & a < 1 \\
  $D_f = \R^+, Z_f = \R$, & $D_f = \R^+, Z_f = \R$, \\
  ničla $x = 1$, naraščajoča, konkavna, & ničla $x = 1$, padajoča, konveksna, \\
  pozitivna $x > 1$, negativna $x < 1$, & pozitivna $x < 1$, negativna $x > 1$, \\
  navzgor omejena & navzgor omejena \\
  graf na sliki~\ref{fig:fun:log:avec} & graf na sliki~\ref{fig:fun:log:amanj}
\end{tabular}

\begin{figure}[ht]
  \begin{center}
    \psset{unit=.5cm}
    \subfigure[a > 1]{
      \label{fig:fun:log:avec}
      \begin{pspicture*}(-4.9,-4.9)(4.9,4.9)
        \psaxes[labels=none]{->}(0,0)(-4.9,-4.9)(4.9,4.9)
        \psplot[plotstyle=curve,swapaxes=true, linecolor=red, linewidth=1pt]{-4.9}{4.9}{2.7182 x exp}
        \psplot[plotstyle=curve,swapaxes=true, linecolor=blue, linewidth=1pt]{-4.9}{4.9}{1.5 x exp}
        \psplot[plotstyle=curve,swapaxes=true, linecolor=green, linewidth=1pt]{-4.9}{4.9}{10 x exp}
        \uput[-90](3.7,0){$\log_{10}\!x$}
        \uput[0](2.7,2){$\log_e\!x$}
        \uput[0](1.5,3.8){$\log_{1,5}\!x$}
      \end{pspicture*}
    }
    \subfigure[a < 1]{
      \label{fig:fun:log:amanj}
      \begin{pspicture*}(-4.9,-4.9)(4.9,4.9)
        \psaxes[labels=none]{->}(0,0)(-4.9,-4.9)(4.9,4.9)
        \psplot[plotstyle=curve,swapaxes=true, linecolor=red, linewidth=1pt]{-4.9}{4.9}{0.4 x exp}
        \psplot[plotstyle=curve,swapaxes=true, linecolor=blue, linewidth=1pt]{-4.9}{4.9}{0.66666667 x exp}
        \psplot[plotstyle=curve,swapaxes=true, linecolor=green, linewidth=1pt]{-4.9}{4.9}{0.1 x exp}
        \uput[90](3.5,0){$\log_{0,1}\!x$}
        \uput[0](2.2,-2.2){$\log_{0,4}\!x$}
        \uput[0](2,-4){$\log_{\frac{2}{3}}\!x$}
      \end{pspicture*}
    }
    \psset{unit=1cm}
  \end{center}
  \beforecaptionskip
  \caption{Graf logaritemske funkcije.}
  \label{fig:fun:log:graf}
\end{figure}

Navpična \textbf{asimptota} je $y$ os. Vse logaritemske funkcije gredo skozi točko $N(1,0)$, kar
izhaja iz izpeljave iz definicije~\eqref{eq:log:log1je0}. Vse funkcije z bazo iz enake 
skupine se razlikujejo le
po \textbf{strmini} padanja in naraščanja. Lahko jih premikamo ali raztegujemo.
\[ f(x) = b\krat\log_a\!(x-p) + q \]

\subsection{Krožne funkcije}
\label{sec:fun:arc}
\textbf{Krožne funkcije} ali \textbf{arcus funkcije} so \textbf{delni inverzi} kotnih funkcij.\footnote{Kotne funkcije
so definirane kasneje, v razdelku \ref{sec:kot}.}

\textbf{Arcus sinus} $x$ je tisti kot, pri katerem je sinus enak $x$.
\[ y = \arcsin x \iff \sin y = x, \; D_f = [-1,1], Z_f = \left[ -\frac{\pi}{2},
\frac{\pi}{2} \right] \]

\textbf{Arcus kosinus} $x$ je tisti kot, pri katerem je kosinus enak $x$.
\[ y = \arccos x \iff \cos y = x, \; D_f = [-1,1], Z_f = \left[0, \pi \right] \]

\textbf{Arcus tangens} $x$ je tisti kot, pri katerem je tangens enak $x$.
\[ y = \arctan x \iff \tan y = x, \; D_f = \R, Z_f = \left( -\frac{\pi}{2},
\frac{\pi}{2} \right) \]

\textbf{Arcus kotangens} $x$ je tisti kot, pri katerem je kotangens enak $x$.
\[ y = \arccot x \iff \cot y = x, \; D_f = \R, Z_f = \left(0, \pi \right) \]

Grafi arcus funkcij so prikazani na sliki~\ref{fig:fun:arc:grafi}.

\begin{figure}[ht]
  \begin{center}
    \psset{unit=1.1cm}
    \subfigure[Arcus sinus]{
      \label{fig:fun:arc:sin}
      \begin{pspicture*}(-2.5,-2.5)(2.5,2.5)
        \psaxes[labels=none]{->}(0,0)(-2.5,-2.5)(2.5,2.5)
        \psplot[plotstyle=curve,linecolor=black, linewidth=1pt,swapaxes=true]{-1.5}{1.5}{x 60 mul sin}
        \oznaka(0,1.5)(1,1.5)
        \oznaka(0,-1.5)(-1,-1.5)
        \oznaka(-1,0)(-1,-1.5)
        \oznaka(1,0)(1,1.5)
        \uput[180](0,1.5){$\frac{\pi}{2}$}
        \uput[0](0,-1.5){$-\frac{\pi}{2}$}
        \uput[-90](1,0){$1$}
        \uput[90](-1,0){$-1$}
      \end{pspicture*}
    }
    \subfigure[Arcus kosinus]{
      \label{fig:fun:arc:cos}
      \begin{pspicture*}(-2.5,-2.5)(2.5,2.5)
        \psaxes[labels=none]{->}(0,-1)(-2.5,-2.5)(2.5,2.5)
        \psplot[plotstyle=curve,linecolor=black, linewidth=1pt,swapaxes=true]{-1}{2}{x 1 add 60 mul cos}
        \oznaka(-1,-1)(-1,2)
        \oznaka(-1,2)(0,2)
        \uput[0](0,2){$\pi$}
        \uput[-90](1,-1){$1$}
        \uput[-90](-1,-1){$-1$}
      \end{pspicture*}
    } \\
    \subfigure[Arcus tangens]{
      \label{fig:fun:arc:tan}
      \begin{pspicture*}(-2.5,-2.5)(2.5,2.5)
        \psaxes[labels=none]{->}(0,0)(-2.5,-2.5)(2.5,2.5)
        \psplot[plotstyle=curve,linecolor=black, linewidth=1pt,swapaxes=true]{-1.4}{1.4}{x 60 mul sin x 60 mul cos div}
        \asimptota(-3,1.5)(3,1.5)
        \asimptota(-3,-1.5)(3,-1.5)
        \uput[180](0,1.5){$\frac{\pi}{2}$}
        \uput[0](0,-1.5){$-\frac{\pi}{2}$}
      \end{pspicture*}
    }
    \subfigure[Arcus kotangens]{
      \label{fig:fun:arc:cot}
      \begin{pspicture*}(-2.5,-2.5)(2.5,2.5)
        \psaxes[labels=none]{->}(0,-1)(-2.5,-2.5)(2.5,2.5)
        \psplot[plotstyle=curve,linecolor=black, linewidth=1pt,swapaxes=true]{-0.9}{1.9}{x 1 add 60 mul cos
        x 1 add  60 mul sin div}
        \asimptota(-3,2)(3,2)
        \asimptota(-3,-0.95)(3,-0.95)
        \uput[45](0,2){$\pi$}
      \end{pspicture*}
    }
    \psset{unit=1cm}
  \end{center}
  \beforecaptionskip
  \caption{Grafi arcus funkkcij.}
  \label{fig:fun:arc:grafi}
\end{figure}

\subsection{Racionalne funkcije}
\label{sec:fun:rac}
\textbf{Racionalna funkcija} je vsaka funkcija oblike $f(x) = \frac{p(x)}{q(x)}$, pri čemer je ta
ulomek okrajšan.

\textbf{Ničle} racionalne funkcije so ničle polinoma $p(x)$, \textbf{poli} racionalne funkcije pa so ničle
polinoma $q(x)$. \textbf{Stopnja} pola racionalne funkcije je enaka stopnji ničle
imenovalca. Stopnja ničle racionalne funkcije je enaka stopnji ničle imenovalca.

Pri polih in ničlah \textbf{lihe} stopnje se predznak racionalne funkcije spremeni, pri polih ali
ničlah \textbf{sode} stopnje pa se ohrani. Bližje kot smo polu, večje so funkcijske
vrednosti po absolutni vrednosti.

Vsako racionalno funkcijo lahko zapišemo kot vsoto polinoma in nove racionalne funkcije,
ki ima v števcu polinom nižje stopnje kot v imenovalcu.
\begin{align*}
    p(x) &= k(x) \krat q(x) + o(x) \; /:q(x)  \quad st(q(x)) > st(o(x)) \comment{osn. izr. o
    del. pol.} \\
    \frac{p(x)}{q(x)} &= k(x) + \frac{o(x)}{q(x)} 
\end{align*}

$k(x)$ je \textbf{asimptota} racionalne funkcije. Je krivulja, kateri se graf približuje pri zelo
velikih in majhnih $x$-ih, ker je takrat ulomek $\frac{o(x)}{q(x)} \approx 0$, ker je $q(x)
\gg o(x)$.

\textbf{Presečišče z asimptoto} (kadar je funkcijska vrednost enaka $k(x)$)
\begin{align*}
    \frac{p(x)}{q(x)} &= k(x) + \frac{o(x)}{q(x)} \\
    f(x) &= k(x) \iff \frac{o(x)}{q(x)} = 0 \iff o(x) = 0 \\
\end{align*}

\section{Enačbe}
\label{sec:enac}
\textbf{Enačba}\index{Enačba} je zapis za enakost dveh izrazov. Izraza imenujemo \textbf{leva stran}
enačbe in \textbf{desna stran}
enačbe. Med njima stoji \textbf{enačaj}. Spremenljivke, ki nastopajo v enačbi, imenujemo
\textbf{neznanke}.
Vrednost neznanke, ki zadosti enakosti imenujemo \textbf{rešitev} ali \textbf{koren} enačbe.

Enačbi, ki imata enaki množici rešitev sta enakovredni ali \textbf{ekvivalentni}.

Enačba, ki nima rešitve se imenuje \textbf{nerešljiva enačba}. Primer:
\[ x + 1 = x + 3 \]

Če je enakost enačbe velja ne glede na vrednost neznanke, tako enačbo imenujemo identična
enačba ali \textbf{identiteta}. Primer:
\[2x^2 -(x + 1)^2 - 4 = x^2 - 2x - 5 \]

\subsection{Reševanje enačb}
\label{sec:enac:resev}
Enačbo lahko preoblikujemo v drugo ekvivalentno enačbo z naslednjimi postopki:
\begin{itemize*}
  \item Levo ali desno stran enačbe lahko preoblikujemo s pravili za preoblikovanje
    izrazov.
  \item Enačbi lahko na obeh straneh \textbf{prištejemo} ali \textbf{odštejemo} poljubno število
ali izraz. Iz tega izhaja tudi ``prenašanje'' člena preko enačaja (na obeh straneh odštejemo ali
prištejemo ta člen).

  \item Enačbo lahko na obeh straneh \textbf{množimo} ali \textbf{delimo} s poljubnim številom ali izrazom, ki ni
enak 0. 

  \item Na levi in na desni strani lahko \textbf{izvedemo} isto matematično \textbf{funkcijo}, ki mora biti
\textbf{bijektivna}.\footnote{Za definicijo bijektivne preslikave glej
razdelek~\ref{sec:preslikave}.}
\end{itemize*}

Pozor: Če levo in desno stran pomnožimo ali delimo z matematičnim izrazom, ki bi lahko bil
enak 0 (za določeno vrednost spremenljivke), dobljena enačba ni nujno enakovredna prvotni.
Če na levi in desni strani izvedemo funkcijo, ki ni bijektivna (npr. kvadriranje),
dobljena enačba ni nujno enakovredna prvotni.

\subsection{Linearne enačbe}
\label{sec:enac:lin}
\textbf{Linearna enačba} je vsaka enačba oblike $kx + n = 0; k, n \in \R$ ali vsaka enačba, ki jo v to
obliko lahko prevedemo.

\textbf{Število rešitev} linearne enačbe:
\begin{align*}
  k \neq 0 &\implies \textnormal{1 rešitev} \\
  k = 0 \land n = 0 &\implies \infty \textnormal{ rešitev, identiteta} \\
  k = 0 \land n \neq 0 &\implies \textnormal{ni rešitve}
\end{align*}

\subsection{Razcepne enačbe}
\label{sec:enac:razc}
\[ A \krat B = 0 \implies A = 0 \lor B = 0 \]
Primer uporabe:
\begin{align*}
  x^2 +5x + 6 &= 0 \\
  (x+3)(x+2) &= 0 \\
  1.\quad x + 3 &= 0 \implies x_1 = -3 \\
  2.\quad x + 2 &= 0 \implies x_2 = -2
\end{align*}

\subsection{Kvadratne enačbe}
\label{sec:enac:kvad}
\textbf{Kvadratna enačba} je vsaka enačba oblike $ax^2 + bx + c = 0; a, b, c \in \R$ in $a \neq 0$
ali vsaka enačba, ki jo v to obliko lahko prevedemo.

Kvadratna enačba oblike $ax^2 + bx + c = 0$ sprašuje po \textbf{ničlah} funkcije $f(x) = ax^2 + bx
+ c$. Za rešitvi enačbe imamo formulo \eqref{eq:fun:kvad:nicle}.

Kvadratne enačba ima:
\begin{itemize*}
  \item dve različni realni rešitvi, če $D > 0$
  \item eno dvojno realno rešitev, če $D = 0$
  \item dve kompleksni\footnote{Kompleksa števila so definirana kasneje, v razdelku
    \ref{sec:kompl}.} rešitvi, ki sta par konjugiranih števil, če $D < 0$.
\end{itemize*}

\subsubsection{Vi\'{e}tovi formuli}
\label{sec:enac:kvad:viet}
Če je pri kvadratni enačbi $a$ enak 1:
\begin{align*}
  x&^2 + ux + v = 0 \\
  u& = -(x_1 + x_2) \\
  v& = x_1 \krat x_2
\end{align*}

\subsection{Kompleksne enačbe}
\label{sec:enac:kompl}
\textbf{Kompleksna enačba} je vsaka enačba oblike $z = w; z, w \in \C$ ali vsaka neenačba, ki jo v
to obliko lahko prevedemo.

Kompleksno število je enako nič, če sta obe njegovi komponenti enaki nič.
\[ A + B\ii = 0 \iff A = 0 \land B = 0 \]

Dve kompleksni števili sta enaki, če sta njuni realni in imaginarni komponenti enaki.
\[ A + B\ii = C + D\ii \iff A = C \land B = D \]

\subsection{Eksponentne enačbe}
\label{sec:enac:eks}
\textbf{Eksponentna enačba} je vsaka enačba v kateri neznanka nastopa v eksponentu.

Enostavne rešitve enačbe:
\begin{align*}
  a^x = a^y &\iff x = y \\
  a^x = 1 &\iff x = 0 \\
  a^x = b^x &\iff x = 0
\end{align*}

Poznamo štiri tipe enačb: \\
Primer: $2^{2x+3} = 8$. Rešujemo s pravili zgoraj.\\
Primer: $3^{x+1} - 3^{x-1} = 24$. Reševanje z izpostavljanjem.\\
Primer: $2^x - 2^{2x-1} = 4$. Reševanje s substitucijo.\\
Primer: $4^x = 10$. Reševanje z logaritmiranjem.

\subsection{Logaritemske enačbe}
\label{sec:enac:log}
\textbf{Logaritemska enačba} je vsaka enačba v katerih nastopa neznanka v logaritmu.

Najprej damo vse logaritme na eno osnovo, skrčimo, nato \textbf{antilogaritmiramo} ali razrešimo po
definiciji in rešimo nastalo enačbo. Lahko se rešujejo tudi s \textbf{substitucijo}.

\subsection{Trigonometrične enačbe}
\label{sec:enac:trig}
\textbf{Trigonometrična enačba} je vsaka enačba v kateri nastopa neznanka v \textbf{kotnih funkcijah}.

Enostavne: $\sin x = a; \; a \in \R$. Običajno dve neskončni množici rešitev. Skica je
priporočljiva.

\index{Enačba!Homogena enačba}
\textbf{Homogene:} $A\sin x + B\kos x = 0$ in podobne višjih stopenj. Lahko se deli s $\kos x$ ali
$\sin x$, ker noben izmed njiju ni enak 0. Vsi členi morajo imeti enako število faktorjev
s kotno funkcijo.

Produkt dveh kotnih funkcij je enak 0: $\sin x \krat \tan x = 0$. Glej razcepne enačbe
(razdelek \ref{sec:enac:razc}).

Uporaba \textbf{faktorizacije}, \textbf{substitucije}, metoda \textbf{polovičnih kotov} (substitucija $x = 2\alpha$),
\textbf{razčlenjevanje} (produkt dveh kotnih funkcij v enem členu)

\subsection{Polinomske enačbe}
\label{sec:enac:pol}
\textbf{Polinomska\footnote{Polinomi so definirani kasneje, v razdelku \ref{sec:pol}}
enačba} je vsaka enačba oblike $p(x) = 0$ ali vsaka enačba, ki jo v to obliko lahko prevedemo. 

Rešitve enačbe so ničle polinoma $p(x)$.

\subsection{Racionalne enačbe}
\label{sec:enac:rac}
\textbf{Racionalna enačba} je vsaka enačba oblike $\frac{p(x)}{q(x)} = 0$ ali vsaka enačba, ki jo
v to obliko lahko prevedemo. $p(x)$ in $q(x)$ sta polinoma.
Pomembno je, da si pri reševanju take enačbe zapišemo pogoje
za rešitve (ničle imenovalcev ne smejo biti rešitve).

\section{Neenačbe}
\label{sec:neenac}
\textbf{Neenačba} je simbolični zapis sestavljen iz dveh matematičnih
izrazov, med katerima stoji \textbf{neenačaj}. \hyperanchor{point:neenacaj}
Neenačaj je lahko katerikoli od znakov za relacijo
urejenosti ($<$, $\leq$, $>$, $\geq$, včasih tudi $\neq$). Izraza, ki nastopata v neenačbi,
imenujemo \textbf{leva stran} in \textbf{desna stran} neenačbe.
Spremenljivke, ki nastopajo v neenačbi, imenujemo \textbf{neznanke}.
\textbf{Rešitev} neenačbe je vrednost neznanke, ki zadosti neenakosti. 
Množico rešitev, ki je pogosto neskončna, ponavadi zapišemo z intervalom.
Primer:
\[ x + 1 \leq 2 \implies x \in (-\infty, 1] \]

Neenačbi sta enakovredni ali \textbf{ekvivalentni}, če imata enako množico rešitev.
Primer:
\[ 3x + 1 < x + 7 \textnormal{ in } 2x < 6 \]

Neenačbe se v nalogah dostikrat povezuje z definicijskim območjem funkcij. Primer:\\
Poišči definicijsko območje funkcije $f(x) = \log(x^3 + 2x - 4)$ je enako kot: reši
neenačbo $x^3 + 2x - 4 >0$.

\subsection{Reševanje neenačb}
\label{sec:neenac:resev}
Neenačbo lahko preoblikujemo v drugo ekvivalentno neenačbo z naslednjimi postopki:
\begin{itemize*}
  \item Levo ali desno stran neenačbe lahko preoblikujemo s pravili za preoblikovanje
    izrazov.
  \item Neenačbi lahko na desni in na levi strani \textbf{prištejemo} ali \textbf{odštejemo} isto število ali izraz.
Prav tako lahko tudi ``prenesemo'' člene preko neenačaja, tako da jim spremenimo predznak.

  \item Neenačbo lahko na desni in na levi strani \textbf{množimo} ali \textbf{delimo} z istim \textit{pozitivnim} številom
ali izrazom.

  \item Če neenačbo na desni in na levi strani \textbf{množimo} ali \textbf{delimo} z istim \textit{negativnim} številom
ali izrazom, se neenačaj obrne.

  \item Na levi in desni strani lahko \textbf{izvedemo} isto matematično \textbf{funkcijo}, ki pa mora biti povsod 
\textit{strogo rastoča}.

  \item Če na levi in desni strani \textbf{izvedemo} isto matematično \textbf{funkcijo}, ki je povsod \textit{strogo
padajoča}, se neenačaj obrne
\end{itemize*}

Rešitev \textbf{sistema} neenačb je presek rešitev posameznih neenačb.

\subsection{Linearne neenačbe}
\label{sec:neenac:lin}
\textbf{Linearna neenačba} je vsaka neenačba oblike $kx + n$
\hyperlink{point:neenacaj}{\texttt{neenačaj}} $0; k, n \in \R$ ali vsaka
neenačba, ki jo v to obliko lahko prevedemo.

\subsection{Kvadratne neenačbe}
\label{sec:neenac:kvad}
\textbf{Kvadratna neenačba} je vsaka neenačba oblike $ax^2 + bx + c$
\hyperlink{point:neenacaj}{\texttt{neenačaj}} $0$
ali vsaka neenačba, ki jo v to obliko lahko prevedemo.

Rešitve poiščemo tako, da izračunamo ničle funkcije $f(x)
= ax^2 + bx + c$ in ugotovimo predznak kvadratne funkcije na celotni realni osi ter nato 
izberemo želene intervale, ki ustrezajo pogojem. Skica je priporočljiva.

\subsection{Polinomske neenačbe}
\label{sec:neenac:pol}
Polinomska neenačba je vsaka neenačba oblike $p(x)$
\hyperlink{point:neenacaj}{\texttt{neenačaj}} $0$ ali vsaka
neenačba, ki jo v to obliko lahko prevedemo.

Rešitve poiščemo tako, da izračunamo ničle polinoma $p(x)$
in ugotovimo predznak funkcije na celotni realni osi ter nato 
izberemo želene intervale, ki ustrezajo pogojem. Skica je priporočljiva.

\subsection{Racionalne neenačbe}
\label{sec:neenac:rac}
Racionalna neenačba je vsaka neenačba oblike $\frac{p(x)}{q(x)}$
\hyperlink{point:neenacaj}{\texttt{neenačaj}} $0$ ali
vsaka neenačba, ki jo v to obliko lahko prevedemo. Rešimo jo
tako da vse člene prenesemo na eno stran, in določimo ničle in pole dobljene racionalne
funkcije, ter tako ugotovimo njen predznak na celotni realni
osi in nato izberemo želeni interval kot rešitev neenačbe. Skica je priporočljiva.

\section{Geometrija}
\label{sec:geom}
Listi!\\
Naslednje dokaze je treba znat:
\begin{enumerate*}
  \item vsota notranjih kotov v trikotniku: $\alpha + \beta + \gamma = 180\deg$, grafično
  \item vsota zunanjih kotov v trikotniku: $\alpha' + \beta' + \gamma' = 360\deg$, grafično in računsko
  \item zveza med zunanjimi in notranjimi koti: $\alpha' = \beta + \gamma$, grafično in
    računsko \label{enum:geom:notrzun}
  \item središčni in obodni kot, grafično
  \item Talesov izrek: kot ki ima vrh na krožnici, kraka pa potekata skozi krajišči
    polmera, meri 90\deg.
\end{enumerate*}

\section{Podobnost}
\label{sec:podob}
Enakoležne stranice so tiste, ki ležijo nasproti istim kotom.

\subsection{Talesovi izreki}
\label{sec:podob:tales}
Če sta si trikotnika podobna, je razmerje dveh enakoležnih stranic enako razmerju drugih
dveh enakoležnih stranic.
\begin{equation}
  \frac{a_1}{a} = \frac{b_1}{b} = \frac{c_1}{c} = k 
  \label{eq:podob:tal1}
\end{equation}

Če sta si trikotnika podobna, je razmerje stranic prvega trikotnika enako razmerju
enakoležnih stranic drugega trikotnika.
\begin{equation}
  a : b : c = a_1 : b_1 : c_1
  \label{eq:podob:tal2}
\end{equation}

Če se trikotnika ujemata v kotu in razmerju stranic, ki kot oklepata, sta si podobna.

Razmerje obsegov, višin in ploščin:
\[ \frac{o_1}{o} = k  \\
 \frac{v_1}{v} = k  \\
 \frac{p_1}{p} = k^2 \]

\subsection{Izreki v pravokotnem trikotniku}
\label{sec:podob:prav}

Višinski izrek:
\begin{equation}
  v_c^2 = a_1 \krat b_1
  \label{eq:podob:trik:vis}
\end{equation}

Evklidov izrek:
\begin{equation}
  a^2 = a_1 \krat c \\
  b^2 = b_1 \krat c
  \label{eq:podob:trik:evk}
\end{equation}

Pitagorov izrek:
\begin{equation}
  c^2 = a^2 + b^2
  \label{eq:podob:trik:pit}
\end{equation}

Dokaz:
\begin{align*}
  c^2 &= a^2 + b^2 \\
  c^2 &= a_1 \krat c + b_1 \krat c \\
  c^2 &= c \krat (a_1 + b_1) \\
  c^2 &= c \krat c \\
\end{align*}

%\begin{figure}[ht]
%  \begin{center}
%    \begin{pspicture*}(-4.9,-4.9)(4.9,4.9)
%      \pstTriangle(-3,0){A}(1.5,2.6){C}(3,0){B}
%      \pstRightAngle{A}{C}{B}
%      \pstProjection{A}{B}{C}[T]
%      \pstLineAB{C}{T}[$v_c$]
%    \end{pspicture*}
%  \end{center}
%  \beforecaptionskip
%  \caption{Višinski in Evklidov izrek v trikotniku.}
%  \label{fig:podob:trik:izr}
%\end{figure}


\section{Kotne funkcije}
\label{sec:kot}

\subsection{V pravokotnem trikotniku}
\label{sec:kot:prav}
\textbf{Sinus kota} je enak razmerju med kotu nasprotno kateto in hipotenuzo.
\[ \sin\alpha = \frac{a}{c} \]
\textbf{Kosinus kota} je enak razmerju med kotu priležno kateto in hipotenuzo.
\[ \cos\alpha = \frac{a}{c} \]
\textbf{Tangens kota} je enak razmerju med kotu nasprotno in kotu priležno kateto.
\[ \tan\alpha = \tg\alpha = \frac{a}{c} \]
\textbf{Kotangens kota} je enak razmerju med kotu priležno in kotu nasprotno kateto.
\[ \cot\alpha = \ctg\alpha = \frac{a}{c} \]

\begin{table}[h]
  \centering
  \caption{Vrednosti kotnih funkcij za določene kote.}
  \label{tab:kot:prav:vred}
  \begin{tabularx}{\textwidth}{C!{\vrule width 1.5pt}C|C|C|C}
     $\alpha$ & $\sin\alpha$ & $\kos\alpha$ & $\tan\alpha$ & $\cot\alpha$ \\ \noalign{\hrule height 1.5pt}
     $0\deg$  &     $0$      &      $1$     &     $0$      &   nedef.     \\ \hline
     $30\deg$ & \mltc{$\frac{1}{2}$} & \mltc{$\frac{\sqrt{3}}{2}$} & \mltc{$\frac{\sqrt{3}}{3}$} & $\sqrt{3}$ \\ \hline
     $45\deg$ & \mltc{$\frac{\sqrt{2}}{2}$} & \mltc{$\frac{\sqrt{2}}{2}$} & $1$ & $1$ \\ \hline
     $60\deg$ & \mltc{$\frac{\sqrt{3}}{2}$} & \mltc{$\frac{1}{2}$} & $\sqrt{3}$ & \mltc{$\frac{\sqrt{3}}{3}$} \\ \hline
     $90\deg$ &     $1$      & \mltc{$\frac{\sqrt{3}}{2}$} & nedef. & $0$ \\
  \end{tabularx}
\end{table}

\subsection{Kot}
\label{sec:kot:kot}
Definicijo kota za delo s kotnimi funkcijami raz\v{s}irimo tako, da kotu dolo\v{c}imo
\textbf{smer} (slika~\ref{fig:kot:kot:plus} in~\ref{fig:kot:kot:minus}), tako da določimo prvi in drugi krak (kot tako vedno merimo od prvega do drugega
kraka po krajši poti) in da dopuščamo \textbf{poljubno velike} kote
(slika~\ref{fig:kot:kot:vel}).

\begin{figure}[ht]
  \begin{center}
    \subfigure[Pozitiven kot.]{
      \label{fig:kot:kot:plus}
      \begin{pspicture*}(0,-1)(3.5,1)
        \psline(0,0)(3,1)
        \psline(0,0)(3,-1)
        \pscurve{<-}(1.5,.5)(1.6,0)(1.5,-.5)
        \uput[d](3,1){2}
        \uput[u](3,-1){1}
        \uput[r](1.6,0){$+$}
        \uput{.7}[r](0,0){$\varphi$}
      \end{pspicture*}
    }
    \subfigure[Negativen kot.]{
      \label{fig:kot:kot:minus}
      \begin{pspicture*}(0,-1)(3.5,1)
        \psline(0,0)(3,1)
        \psline(0,0)(3,-1)
        \pscurve{->}(1.5,.5)(1.6,0)(1.5,-.5)
        \uput[d](3,1){1}
        \uput[u](3,-1){2}
        \uput[r](1.6,0){$-$}
        \uput{.7}[r](0,0){$\varphi$}
      \end{pspicture*}
    }
    \subfigure[Poljubno velik kot.]{
      \label{fig:kot:kot:vel}
      \begin{pspicture*}(-3.5,-3.5)(3.5,3.5)
        \psline(0,0)(3,1)
        \psline(0,0)(3,-1)
        \pscurve{->}(.60,-.2)(0,-.7)(-.8,0)(0,.9)(1.1,0)(0,-1.2)(-1.3,0)(0,1.4)(1.48,.5)
        \uput[l](0,0){$\varphi$}
      \end{pspicture*}
    }
  \end{center}
  \beforecaptionskip
  \caption{Razširjena definicija kota}
  \label{fig:kot:kot}
\end{figure}

Kot tudi merimo v različnih enotah. \textbf{Radian} je enota, ki predstavlja dolžino krožnega loka
z radijem 1 nad določenim kotom. Pretvorba določenih vrednosti iz stopinj v radiane je
prikazana v tabeli~\ref{tab:kot:kot:degrad}.

\begin{table}[h]
  \centering
  \caption{Tabela pretvorb med radiani in stopinjami za določene kote.}
  \vspace{5pt}
  \label{tab:kot:kot:degrad}
  \begin{tabular}{c|c|c|c|c|c|c|c|c|c|c}
    stopinje & $ 0\deg $ & $ 30\deg $ & $ 45\deg $ & $ 60\deg $ & $ 90\deg $ & $ 120\deg $ & $ 135\deg $ & $ 150\deg $ & $ 180\deg $ & $ 360\deg $ \\ \hline
    radiani & \mltc{$ 0     $} & \mltc{$ \frac{\pi}{6}$} & \mltc{$ \frac{\pi}{4}$} &
    \mltc{$ \frac{\pi}{3}$} & \mltc{$ \frac{\pi}{2}$} & \mltc{$ \frac{2\pi}{2}$} &
    \mltc{$ \frac{3\pi}{4}$} & \mltc{$ \frac{5\pi}{6}$} & \mltc{$ \pi$} & \mltc{$ 2\pi$} \\
  \end{tabular}
\end{table}

\subsection{Sinus in kosinus}
\label{sec:kot:sincos}

\begin{figure}[ht]
  \begin{center}
      \psset{unit=2cm}
      \begin{pspicture*}(-1.15,-1.15)(1.15,1.15)
        \psaxes[labels=none]{->}(0,0)(-1.15,-1.15)(1.15,1.15)
        \pscircle(0,0){1}
        \psline[linewidth=1.5pt]{->}(0,0)(0.6,0.8)
        \psline[linewidth=1.5pt]{->}(0,0)(1,0)
        \uput[dl](0,0){$0$}
        \uput[dl](0,1){$1$}
        \uput[dr](1,0){$1$}
        \uput[d](.5,0){$\vec{a}$}
        \uput[120](.3,.5){$\vec{b}$}
        \uput[ur](.2,0){$\alpha$}
        \uput[60](.6,.8){$T\left(x,y\right)$}
      \end{pspicture*}
      \psset{unit=1cm}
  \end{center}
  \beforecaptionskip
  \caption{Kot med enotskima vektorjema, uporabljen pri definiciji sinusa in kosinusa.}
  \label{fig:kot:sincosdef}
\end{figure}
Ob izpeljavi glej sliko~\ref{fig:kot:sincosdef}.
\[ \vec{a} = \left(1,0\right) \]
\[ \vec{b} = \left(x,y\right) \]
\[ \vec{a} \krat \vec{b} = ab\kos\alpha\;\footnotemark
  \comment{$\vec{a}$ in $\vec{b}$ sta enotska vektorja} \]
\begin{equation}
  \kos\alpha = \frac{\vec{a}\krat\vec{b}}{a\krat b} = \vec{a}\krat\vec{b} = \left( 1,0 \right)\krat\left( x,y \right) = 1x + 0y = x
  \label{eq:kot:cosdef}
\end{equation}

\[ \left|\vec{a} \times \vec{b}\right| = \left|\left( 
\begin{vmatrix} a_2 & a_3 \\ b_2 & b_3 \end{vmatrix},
\begin{vmatrix} a_3 & a_1 \\ b_3 & b_1 \end{vmatrix},
\begin{vmatrix} a_1 & a_2 \\ b_1 & b_2 \end{vmatrix}
\right)\right| =  \left| \left( 0,0, \begin{vmatrix} a_1 & a_2 \\ b_1 & b_2
\end{vmatrix} \right) \right| = \begin{vmatrix} a_1 & a_2 \\ b_1 & b_2 \end{vmatrix} =
ab\sin\alpha \] 

\begin{equation}
  \sin\alpha = \frac{\begin{vmatrix} a_1 & a_2 \\ b_1 & b_2 \end{vmatrix}}{a\krat b} =
  \begin{vmatrix} a_1 & a_2 \\ b_1 & b_2 \end{vmatrix} =
  \begin{vmatrix} 1 & 0 \\ x & y \end{vmatrix} = 1y - 0x = y
  \label{eq:kot:sindef}
\end{equation}

\footnotetext{Za formule, ki se tičejo vektorjev glej
razdelek~\ref{sec:vec}. Za skalarni produkt glej razdelek~\ref{sec:vec:skal}, za vektorski
produkt pa~\ref{sec:vec:vec}.}

\textbf{Sinus} kota, ki ima en krak na pozitivni strani x osi in vrh v izhodišču je
\textbf{abscisa} točke v kateri drugi krat seka enotsko krožnico.

\textbf{Kosinus} kota, ki ima en krak na pozitivni strani x osi in vrh v izhodišču je
\textbf{ordinata} točke v kateri drugi krat seka enotsko krožnico.

Grafična predstavitev sinusa in kosinusa je prikazana na sliki~\ref{fig:kot:daljice}.

\textbf{Lastnosti:}
\begin{enumerate*}
  \item $D_{sin} = D_{cos} = \R$
  \item $Z_{sin} = Z_{cos} = [-1,1]$
  \item Obe sta omejeni $m = -1$, $M = 1$
  \item Sinus je \textbf{liha} funkcija: $\sin(-x) = -\sin(x)$
  \item Kosinus je \textbf{soda} funkcija: $\kos(-x) = \kos(x)$
\end{enumerate*}

\begin{figure}[ht]
  \begin{center}
      \psset{unit=3cm}
      \begin{pspicture*}(-1.15,-1.15)(1.4,1.15)
        \psaxes[labels=none]{->}(0,0)(-1.15,-1.15)(1.15,1.15)
        \pscircle(0,0){1}
        \psline(0,0)(1.4,1.05) % kot = 36.87
        \psline(0,0.6)(0.8,0.6) % y = sin(kot)
        \psline[linewidth=1.5pt,linecolor=red](0,0)(0,0.6) % y = sin(kot)
        \psline(0.8,0)(0.8,0.6) % x = cos(kot)
        \psline[linewidth=1.5pt,linecolor=blue](0,0)(0.8,0) % x = cos(kot)
        \psline(-1.15,1)(1.4,1)
        \psline[linewidth=1.5pt,linecolor=green](1,0)(1,0.75) % y = tan(kot)
        \psline(1,-1.15)(1,1.15)
        \psline[linewidth=1.5pt,linecolor=orange](0,1)(1.33333333333333,1) % x = cot(kot)
        \uput[dl](0,0){$0$}
        \uput[ul](0,1){$1$}
        \uput[dr](1,0){$1$}
        \uput[18](.2,.07){$\varphi$}
        \uput[l](0,.3){$\sin\varphi$}
        \uput[d](.5,0){$\kos\varphi$}
        \uput[u](.5,1){$\cot\varphi$}
        \uput[r](1,.5){$\tan\varphi$}
      \end{pspicture*}
      \psset{unit=1cm}
  \end{center}
  \beforecaptionskip
  \caption{Grafični prikaz vrednosti kotnih funkcij.}
  \label{fig:kot:daljice}
\end{figure}

\subsection{Tangens in kotangens}
\label{sec:kot:tancot}
\textbf{Tangens} kota je enak razmerju med sinusom in kosinusom kota.
\begin{equation}
  \tan\alpha = \frac{\sin\alpha}{\kos\alpha} 
  \label{eq:kot:tandef}
\end{equation}

\textbf{Kotangens} kota je enak razmerju med kosinusom in sinusom kota.
\begin{equation}
  \cot\alpha = \frac{\kos\alpha}{\sin\alpha}
  \label{eq:kot:cotdef}
\end{equation}


\textbf{Tangens} kota je \textbf{ordinata} točke v katerem drugi krak kota ali njegova nosilka seka tangento
na enotsko krožnico v točki $(0,1)$.

\textbf{Kotangens} kota je \textbf{abscisa} točke v katerem drugi krak kota ali njegova nosilka seka
tangento na enotsko krožnico v točki $(1,0)$.

Grafična predstavitev tangensa in kotangensa je prikazana na sliki~\ref{fig:kot:daljice}.

\textbf{Lastnosti:}
\begin{enumerate*}
  \item $D_tan = \R - \left\{ \frac{\pi}{2} + k\pi; \; k \in \Z \right\}$
  \item $D_cot = \R - \left\{ k\pi; \; k \in \Z \right\}$
  \item $Z_tan = Z_cot = \R$
  \item Tangens in kotangens \textbf{sta} lihi funkciji. \[ \tan(-x) = -\tan(x) \;\;
    \cot(-x) = -\cot(x) \]
  \item Obe funkciji sta \textbf{periodični} s periodo $\omega = \pi$.
\end{enumerate*}

\subsection{Osnovne zveze med kotnimi funkcijami}
\label{sec:kot:zvez}
\begin{align}
  \sin^2\alpha + \kos^2\alpha &= 1 \label{eq:kot:zvez:sincos} \comment{Grafičen dokaz na
  sliki~\ref{fig:kot:daljice}.} \\
  \tan\alpha \krat \cot\alpha &= 1 \nonumber \\ %\label{eq:kot:zvez:tancot} \\
  1 + \tan^2\alpha &= \frac{1}{\kos^2\alpha} \nonumber \\ %\label{eq:kot:zvez:costan} \\
  1 + \cot^2\alpha &= \frac{1}{\sin^2\alpha} \nonumber \\ %\label{eq:kot:zvez:sincot} \\
\end{align}
Ostale zveze se dokaže tako, da se tangens ali kotangens zamenja po
definiciji~\eqref{eq:kot:tandef} ali~\eqref{eq:kot:cotdef} in nato poenostavi enačbo.

\subsection{Adicijski izreki}
\label{sec:kot:adic}

\begin{figure}[ht]
  \begin{center}
      \psset{unit=2cm}
      \begin{pspicture*}(-1.15,-1.15)(1.15,1.15)
        \psaxes[labels=none]{->}(0,0)(-1.15,-1.15)(1.15,1.15)
        \pscircle(0,0){1}
        \psline[linewidth=1.5pt]{->}(0,0)(.6,.8)
        \psline[linewidth=1.5pt]{->}(0,0)(.8,-.6)
        \uput[dl](0,0){$0$}
        \uput[dl](0,1){$1$}
        \uput[dr](1,0){$1$}
        \uput[d](.35,-.3){$\vec{a}$}
        \uput[120](.3,.5){$\vec{b}$}
        \uput[ur](.2,0){$\beta$}
        \uput[dr](.2,-.015){$\alpha$}
      \end{pspicture*}
      \psset{unit=1cm}
  \end{center}
  \beforecaptionskip
  \caption{Adicijski izreki.}
  \label{fig:kot:adic}
\end{figure}

Ob izpeljavi glej sliko~\ref{fig:kot:adic}. Izpeljane so iz definiciji kotnih funkcij
sinus~\eqref{eq:kot:sindef}, kosinus~\eqref{eq:kot:cosdef}, tangens~\eqref{eq:kot:tandef} in
kotangens~\eqref{eq:kot:cotdef}.
\[\kos(\alpha+\beta) = \vec{a}\krat\vec{b} = (\kos\alpha, -\sin\alpha)\krat(\kos\beta,
\sin\beta) = \kos\alpha\kos\beta - \sin\alpha\sin\beta \]
\[\sin(\alpha+\beta) = \begin{vmatrix} a_1 & a_2 \\ b_1 & b_2 \end{vmatrix} = 
  \begin{vmatrix*}[r] \kos\alpha & -\sin\alpha \\ \kos\beta & \sin\beta \end{vmatrix*} =
  \sin\alpha\kos\beta - \sin\beta\kos\alpha \]

\begin{align*} \tan(\alpha+\beta) &= \frac{\sin(\alpha+\beta)}{\kos\alpha+\beta} =
\frac{\sin\alpha\kos\beta+\sin\beta\kos\alpha :(\kos\alpha\kos\beta)}{\kos\alpha\kos\beta
- \sin\alpha\sin\beta :(\kos\alpha\kos\beta)} =\\&=
\frac{\frac{\sin\alpha\kos\beta}{\kos\alpha\kos\beta}+\frac{\sin\beta\kos\alpha}{\kos\alpha\kos\beta}}
{\frac{\kos\alpha\kos\beta}{\kos\alpha\kos\beta}+\frac{\sin\alpha\sin\beta}{\kos\alpha\kos\beta}}
= \frac{\tan\alpha+\tan\beta}{1-\tan\alpha\tan\beta} \end{align*}

\begin{align*} \cot(\alpha+\beta) &= \frac{\kos(\alpha+\beta)}{\sin\alpha+\beta} =
\frac{\kos\alpha\kos\beta-\sin\alpha\sin\beta :(\sin\alpha\sin\beta)}{\sin\alpha\kos\beta
+ \sin\beta\kos\alpha :(\sin\alpha\sin\beta)} =\\&=
\frac{\frac{\kos\alpha\kos\beta}{\sin\alpha\sin\beta}+\frac{\sin\alpha\sin\beta}{\sin\alpha\sin\beta}}
{\frac{\sin\alpha\kos\beta}{\sin\alpha\sin\beta}+\frac{\sin\beta\kos\alpha}{\sin\alpha\sin\beta}}
= \frac{\cot\alpha\cot\beta-1}{\cot\alpha+\cot\beta} \end{align*}

\boldmath
\begin{align}
  \kos(\alpha\pm\beta) &= \kos\alpha\kos\beta \mp \sin\alpha\sin\beta \label{eq:kot:adic:cos} \\
  \sin(\alpha\pm\beta) &= \sin\alpha\kos\beta \pm \sin\beta\kos\alpha \label{eq:kot:adic:sin} \\
  \tan(\alpha\pm\beta) &= \frac{\tan\alpha \mp \tan\beta}{1 \pm \tan\alpha\tan\beta} \label{eq:kot:adic:tan} \\
  \cot(\alpha\pm\beta) &= \frac{\cot\alpha\cot\beta \mp 1}{\cot\alpha \mp \cot\beta} \label{eq:kot:adic:cot}
\end{align}
\unboldmath

\subsection{Dvojni koti}
Formule se izpelje iz adicijskih izrekov definiranih v razdelku~\ref{sec:kot:adic}.
\begin{align}
  \sin2x &= 2\sin{x}\kos{x} \label{eq:kot:dblsin} \\
  \kos2x &= \kos^2x - \sin^2x \label{eq:kot:dblcos} \\
  \tan2x &= \frac{2\tan{x}}{1-\tan^2x} \nonumber \\
  \cot2x &= \frac{\cot^2x-1}{2\cot{x}} \nonumber
\end{align}

\subsection{Polovični koti}
\label{sec:kot:polov}

\begin{align}
  \sin\alpha &= 2\sin\frac{\alpha}{2}\kos\frac{\alpha}{2} 
  \comment{Po formuli \eqref{eq:kot:dblsin}.} \label{eq:kot:pol:1} \\
  \kos\alpha &= \kos^2\frac{\alpha}{2} - \sin^2\frac{\alpha}{2}
  \comment{Po formuli \eqref{eq:kot:dblcos}.} \label{eq:kot:pol:2} \\
  1 &= \kos^2\frac{\alpha}{2} + \sin^2\frac{\alpha}{2} 
  \comment{Po osnovni zvezi \eqref{eq:kot:zvez:sincos}.} \label{eq:kot:pol:3}
\end{align}

Odštejemo enačbi \eqref{eq:kot:pol:3} in \eqref{eq:kot:pol:2} med seboj.
\begin{align*}
  1 - \kos\alpha &= 2\sin^2\frac{\alpha}{2} \\
  \sin\frac{\alpha}{2} &= \pm\sqrt{\frac{1-\kos\alpha}{2}}
\end{align*}

Seštejemo enačbi \eqref{eq:kot:pol:3} in \eqref{eq:kot:pol:1} med seboj.
\begin{align*}
  1 + \kos\alpha &= 2\kos^2\frac{\alpha}{2} \\
  \kos\frac{\alpha}{2} &= \pm\sqrt{\frac{1+\kos\alpha}{2}}
\end{align*}

\begin{align*}
  \tan\frac{\alpha}{2} &= \pm\sqrt{\frac{1-\kos\alpha}{1+\kos\alpha}} =
  \frac{\sin\alpha}{1+\kos\alpha} = \frac{1-\kos\alpha}{\sin\alpha} \\
  \cot\frac{\alpha}{2} &= \pm\sqrt{\frac{1+\kos\alpha}{1-\kos\alpha}} =
  \frac{\sin\alpha}{1-\kos\alpha} = \frac{1+\kos\alpha}{\sin\alpha} \\
\end{align*}

\parbox[t]{0.5\textwidth}{\subsection{Komplementarni koti}
\label{sec:kot:kompl}
Po adicijskih izrekih velja:
\begin{align*}
  \sin\left(\frac{\pi}{2} - \theta\right) &= \kos\theta \\
  \kos\left(\frac{\pi}{2} - \theta\right) &= \sin\theta \\
  \tan\left(\frac{\pi}{2} - \theta\right) &= \cot\theta \\
  \cot\left(\frac{\pi}{2} - \theta\right) &= \tan\theta \\
\end{align*}}
\parbox[t]{0.5\textwidth}{
\subsection{Suplementarni koti}
\label{sec:kot:supl}
Po adicijskih izrekih velja:
\begin{align*}
  \sin\left(\pi - \theta\right) &= \sin\theta \\
  \kos\left(\pi - \theta\right) &= -\kos\theta \\
  \tan\left(\pi - \theta\right) &= -\tan\theta \\
  \cot\left(\pi - \theta\right) &= -\cot\theta \\
\end{align*}}

\subsection{Periode}
\label{sec:kot:periode}
Za definicijo periodične funkcije glej razdelek~\ref{sec:fun}.
\begin{align*}
  &\sin(\theta+2k\pi) = \sin\theta; \; k \in \Z \\
  &\kos(\theta+2k\pi) = \cos\theta; \; k \in \Z \\
  &\tan(\theta+k\pi) = \tan\theta; \; k \in \Z \\
  &\cot(\theta+k\pi) = \cot\theta; \; k \in \Z
\end{align*}
\begin{align*}
  &\sin(\theta+k\pi) = (-1)^k\sin\theta; \; k \in \Z \\
  &\kos(\theta+k\pi) = (-1)^k\cos\theta; \; k \in \Z \\
\end{align*}

\subsection{Faktorizacija}
\label{sec:kot:fakt}
\begin{align*}
  x=\alpha+\beta,& \;\; y=\alpha-\beta \\
  \alpha = \frac{x+y}{2},& \;\; \beta=\frac{x-y}{2}
\end{align*}
\begin{align}
  \sin{x}+\sin{y} & = \sin(\alpha+\beta) + \sin(\alpha-\beta) = \label{eq:kot:fakt:1} \\
  &=\sin\alpha\kos\beta+\sin\beta\kos\alpha+\sin\alpha\kos\beta-\sin\beta\kos\alpha = \nonumber \\
  &=2\sin\alpha\kos\beta = \label{eq:kot:fakt:2} \\ &= 2\sin\frac{x+y}{2}\kos\frac{x-y}{2} \nonumber
\end{align}

Ostale formule se izpeljejo podobno.
\begin{align*}
  \sin{x} + \sin{y} &= 2\sin\frac{x+y}{2}\cos\frac{x-y}{2} \\
  \sin{x} - \sin{y} &= 2\cos\frac{x+y}{2}\sin\frac{x-y}{2} \\
  \cos{x} + \cos{y} &= 2\cos\frac{x+y}{2}\cos\frac{x-y}{2} \\
  \cos{x} - \cos{y} &=-2\sin\frac{x+y}{2}\sin\frac{x-y}{2}
\end{align*}

\subsection{Antifaktorizacija}
\label{sec:kot:antif}
Pogledamo enačbi~\eqref{eq:kot:fakt:1} in~\eqref{eq:kot:fakt:2} pri faktorizaciji
(razdelek~\ref{sec:kot:fakt}) in zapišemo naslednjo enakost:
\[ \sin(\alpha+\beta) + \sin(\alpha-\beta) = 2\sin\alpha\kos\beta \]
in izpeljemo
\[ \sin\alpha\kos\beta = \frac{1}{2}\left[ \sin(\alpha+\beta) + \sin(\alpha-\beta) \right]
\edot \]

Podobno naredimo tudi za ostale formule.
\begin{align*}
  \sin\alpha\cos\beta &= \frac{1}{2}\left[ \sin(\alpha+\beta)+\sin(\alpha-\beta) \right] \\
  \cos\alpha\sin\beta &= \frac{1}{2}\left[ \sin(\alpha+\beta)-\sin(\alpha-\beta) \right] \\
  \cos\alpha\cos\beta &= \frac{1}{2}\left[ \cos(\alpha+\beta)+\cos(\alpha-\beta) \right] \\
  \sin\alpha\sin\beta &= -\frac{1}{2}\left[ \cos(\alpha+\beta)-\cos(\alpha-\beta)  \right] \\
\end{align*}


\subsection{Grafi trigonometričnih funkcij}
\label{sec:kot:graf}
Splošna oblika:
\[ f(x) = A\sin\omega(x-p)+q \; \footnotemark \]
\footnotetext{Seveda je lahko namesto funkcije $\sin$ vstavljena tudi katera koli druga
trigonometrična funkcija.}

$A$ --- amplituda \\
$\omega$ --- krožna frekvenca (koliko valov je na intervalu dolžine $2\pi$) \\
$\vec{v} = (p,q)$ --- vektor premika \\
Grafi vseh funkcij so prikazni na sliki~\ref{fig:kot:graf}.

V naslednjih definicijah velja: $k \in \Z$.

\parbox[t]{0.5\textwidth}{
\textbf{Sinus:} \\[6pt]
\begin{tabular}[h!]{ll}
  ničle:    & $x = k\pi$ \\
  minimumi: & $x = \frac{\pi}{2} + 2k\pi$ \\
  maksimumi:& $x = -\frac{\pi}{2} + 2k\pi$ \\  
  graf:     & slika~\ref{fig:kot:graf:sin}
\end{tabular} \\[12pt]
\textbf{Kosinus:}\\[6pt]
\begin{tabular}[h!]{ll}
  ničle:    & $x = \frac{\pi}{2} + k\pi$ \\
  minimumi: & $x = \pi + 2k\pi$ \\
  maksimumi:& $x = 2k\pi$ \\ 
  graf:     & slika~\ref{fig:kot:graf:cos}
\end{tabular} \\[12pt]
}
\parbox[t]{0.5\textwidth}{
\textbf{Tangens:} \\[6pt]
\begin{tabular}[h!]{ll}
  ničle: & $x = k\pi$ \\
  poli:  & $x = \frac{\pi}{2} + k\pi$ \\
  graf:     & slika~\ref{fig:kot:graf:tan}
\end{tabular} \\[24pt]
\textbf{Kotangens:} \\[6pt]
\begin{tabular}[h!]{ll}
  ničle:  & $x = \frac{\pi}{2} + k\pi$ \\
  poli: & $x = k\pi$ \\
  graf:     & slika~\ref{fig:kot:graf:cot}
\end{tabular} \\[12pt]
}

\begin{figure}[ht]
  \begin{center}
    \subfigure[Graf funkcije $\sin(x)$]{
      \label{fig:kot:graf:sin}
      \begin{pspicture*}(-3.5,-3.5)(3.5,3.5)
        \psaxes[labels=none]{->}(0,0)(-3.5,-3.5)(3.5,3.5)
        \psplot{-3.5}{3.5}{x 90 mul sin}
        \uput[ul](0,0){0}
        \uput[d](2,-.1){$\pi$}
        \uput[-120](-2,-.1){$-\pi$}
        \uput[u](-1,.1){$-\frac{\pi}{2}$}
        \uput[d](1,-.1){$\frac{\pi}{2}$}
        \uput[-105](-3,-.1){$-\frac{3\pi}{2}$}
        \uput[u](3,.1){$\frac{3\pi}{2}$}
        \uput[ul](0,1){1}
        \uput[r](0,-1){$-1$}
        \oznaka(0,-1)(-1,-1)
        \oznaka(0,1)(1,1)
        \oznaka(1,0)(1,1)
        \oznaka(-1,0)(-1,-1)
        \oznaka(-3,0)(-3,1)
        \oznaka(3,0)(3,-1)
        \oznaka(0,1)(-3,1)
      \end{pspicture*}
    }
    \subfigure[Graf funkcije $\cos(x)$]{
      \label{fig:kot:graf:cos}
      \begin{pspicture*}(-3.5,-3.5)(3.5,3.5)
        \psaxes[labels=none]{->}(0,0)(-3.5,-3.5)(3.5,3.5)
        \psplot{-3.5}{3.5}{x 90 mul cos}
        \uput[dl](0,0){0}
        \uput[u](2,.05){$\pi$}
        \uput[u](-2,.05){$-\pi$}
        \uput[-80](-1,-.1){$-\frac{\pi}{2}$}
        \uput[d](1,-.1){$\frac{\pi}{2}$}
        \uput[-115](-3,-.1){$-\frac{3\pi}{2}$}
        \uput[-75](3,-.1){$\frac{3\pi}{2}$}
        \uput[ul](0,1){1}
        \uput[ur](0,-1){$-1$}
        \oznaka(0,-1)(-2,-1)
        \oznaka(0,-1)(2,-1)
        \oznaka(2,0)(2,-1)
        \oznaka(-2,0)(-2,-1)
      \end{pspicture*}
    }
    \subfigure[Graf funkcije $\tan(x)$]{
      \label{fig:kot:graf:tan}
      \begin{pspicture*}(-3.5,-3.5)(3.5,3.5)
        \psaxes[labels=none]{->}(0,0)(-3.5,-3.5)(3.5,3.5)
        \psplot{-3.5}{-3.05}{x 90 mul tan}
        \psplot{-2.95}{-1.05}{x 90 mul tan}
        \psplot{-0.95}{0.95}{x 90 mul tan}
        \psplot{1.05}{2.95}{x 90 mul tan}
        \psplot{3.05}{3.5}{x 90 mul tan}
        \asimptota(-3,-3.5)(-3,3.5)
        \asimptota(-1,-3.5)(-1,3.5)
        \asimptota(1,-3.5)(1,3.5)
        \asimptota(3,-3.5)(3,3.5)
        \uput[ul](0,0){0}
        \uput[d](2,-.1){$\pi$}
        \uput[-75](-2,-.1){$-\pi$}
        \uput[-75](-1,-.1){$-\frac{\pi}{2}$}
        \uput[-125](1,-.1){$\frac{\pi}{2}$}
        \uput[-125](3,-.1){$\frac{3\pi}{2}$}
        \uput[-75](-3,-.1){$-\frac{3\pi}{2}$}
        \uput[u](-.7,0){$-\frac{\pi}{4}$}
        \uput[d](.5,-.08){$\frac{\pi}{4}$}
        \uput[l](0,1){1}
        \uput[r](0,-1){$-1$}
        \oznaka(0,1)(.5,1)
        \oznaka(.5,0)(.5,1)
        \oznaka(0,-1)(-.5,-1)
        \oznaka(-.5,0)(-.5,-1)
      \end{pspicture*}
    }
    \subfigure[Graf funkcije $\cot(x)$]{
      \label{fig:kot:graf:cot}
      \begin{pspicture*}(-3.5,-3.5)(3.5,3.5)
        \psaxes[labels=none]{->}(0,0)(-3.5,-3.5)(3.5,3.5)
        \psplot{-3.5}{-2.05}{1 x 90 mul tan div}
        \psplot{-1.95}{-.05}{1 x 90 mul tan div}
        \psplot{.05}{1.95}{1 x 90 mul tan div}
        \psplot{2.05}{3.5}{1 x 90 mul tan div}
        \psplot{-3.5}{-2.05}{1 x 90 mul tan div}
        \asimptota(-2,-3.5)(-2,3.5)
        \asimptota(.03,-3.5)(.03,3.5)
        \asimptota(2,-3.5)(2,3.5)
        \uput[ul](0,0){0}
        \uput[-125](2,-.1){$\pi$}
        \uput[-75](-2,-.1){$-\pi$}
        \uput[-115](-1,-.1){$-\frac{\pi}{2}$}
        \uput[-95](1,-.1){$\frac{\pi}{2}$}
        \uput[-115](3,-.1){$\frac{3\pi}{2}$}
        \uput[-110](-3,-.1){$-\frac{3\pi}{2}$}
        \uput[u](-.7,0){$-\frac{\pi}{4}$}
        \uput[d](.5,-.1){$\frac{\pi}{4}$}
        \uput[l](0,1){1}
        \uput[r](0,-1){$-1$}
        \oznaka(0,1)(.5,1)
        \oznaka(.5,0)(.5,1)
        \oznaka(0,-1)(-.5,-1)
        \oznaka(-.5,0)(-.5,-1)
      \end{pspicture*}
    }
  \end{center}
  \beforecaptionskip
  \caption{Grafi trigonometričnih funkcij}
  \label{fig:kot:graf}
\end{figure}
\subsection{Kot med premicama}
\label{sec:kot:prem}

\textbf{Naklonski kot} premice je pozitiven kot med abscisno osjo in premico.
Če je premica vzporedna abscisni osi je kot enak 0.
\[ k = \frac{y_2-y_1}{x_2-x_1} = \tan\varphi; \; 0\deg \le \varphi < 180\deg \comment{Za
$k$ glej razdelek~\ref{sec:fun:lin}} \]

\begin{figure}[ht]
  \begin{center}
      \begin{pspicture*}(-1.5,-1.5)(3.5,3.5)
        \psaxes[labels=none]{->}(0,0)(-1.5,-1.5)(3.5,3.5)
        \psplot{-1.5}{3.5}{-3 x mul 9 add}
        \psplot{-1.5}{3.5}{x 0.666666 sub}
        \uput[30](1,0){$\alpha_1$}
        \uput[ur](3,0){$\alpha_2$}
        \uput[ur](2,1){$\varphi$}
      \end{pspicture*}
  \end{center}
  \beforecaptionskip
  \caption{Kot med premicama.}
  \label{fig:kot:prem}
\end{figure}

Ob izpeljavi glej sliko~\ref{fig:kot:prem}.
\begin{align}
  k_1 &= \tan\alpha_1 \label{eq:kot:prem:1} \\
  k_2 &= \tan\alpha_2 \label{eq:kot:prem:2}
\end{align}

Po izrekih za kote v trikotniku (razdelek~\ref{sec:geom},
\ref{enum:geom:notrzun}~element seznama) velja:
\begin{align*}
  & \alpha_1 + \varphi = \alpha_2 \\
  & \varphi = \alpha_2 - \alpha_1 \\
  & \tan\alpha = \tan(\alpha_2 - \alpha_1) \\
  & \tan\varphi = \frac{\tan\alpha_2 - \tan\alpha_1}{1+\tan\alpha_1\tan\alpha_2} 
  \comment{Po adicijskem izreku za tangens~\eqref{eq:kot:adic:tan}.} \\
  & \tan\varphi = \left| \frac{k_2-k_1}{1+k_1k_2} \right| \comment{Po
  izpeljavah~\eqref{eq:kot:prem:1} in~\eqref{eq:kot:prem:2}}
\end{align*}

\section{Vektorji}
\label{sec:vec}

\subsection{Skalarni produkt}
\label{sec:vec:skal}

\subsection{Vektorski produkt}
\label{sec:vec:vec}

\section{Kompleksna števila}
\label{sec:kompl}

\section{Polinomi}
\label{sec:pol}

\section{Stožnice}
\label{sec:kroz}
\textbf{Stožnice} so dvorazsežne presečne krivulje, ki nastanejo, če presekamo enojni ali dvojni
neskončni stožec z ravnino pod različnimi koti.

Možni preseki:
\begin{itemize*}
  \item krožnica
  \item elipsa
  \item parabola
  \item hiperbola
  \item dve vzporednici
  \item dve nevzporedni premici
  \item ena premica
  \item točka
  \item prazna množica
\end{itemize*}

Splošna enačba stožnice:
\begin{equation}
  Ax^2 + Bxy + Cy^2 + Dx + Ey + F = 0; \quad A, B, C, D, E, F \in \R
  \label{eq:stoz:def}
\end{equation}

\subsection{Krožnica}
\label{sec:stoz:kroz}
\textbf{Krožnica} je množica točk v ravnini, ki so enako oddaljene od izbrane točke $S$. $S$
imenujemo \textbf{središče} krožnice. Razdaljo med točko krožnice in središčem imenujemo
\textbf{polmer} in
ga označimo z $r$.
\[ \mathcal{K} = \left\{ T(x, y); \; d(T, S) = r \right\} \]

Vstavi izpeljavo enačbe tukaj. Sliko tudi.

Enačba krožnice v \textbf{središčni} legi:
\[ x^2 + y^2 = r^2 \]

\textbf{Eksplicitna} enačba krožnice:
\begin{equation}
    y = \pm \sqrt{r^2 - x^2}
  \label{eq:kroz:eksp}
\end{equation}

Enačba krožnice v \textbf{premaknjeni} legi:
\[ (x-p)^2 + (y-q)^2 = r^2 \]

Potreben pogoj za krožnico. Do njega pridemo tako, da enačbo krožnice v premaknjeni legi
razvijemo do konca in nato končno enačbo primerjamo s splošno enačbo stožnice.
\[ A = C \land B = 0 \comment{konstante so iz splošne enačbe stožnice \eqref{eq:stoz:def}} \]

Krožnici sta \textbf{koncentrični}, če imata skupno središče.

\subsection{Elipsa}
\label{sec:elips}
\textbf{Elipsa} je množica točk v ravnini, ki imajo konstantno vsoto razdalj do dveh izbranih točk,
ki ju imenujemo gorišči. $|G_1G_2| = 2e, r_1 + r_2 = 2a$

Vstavi izpeljavo enačbe tukaj. Sliko tudi.

Enačba elipse v \textbf{središčni} legi:
\[ a^2b^2 = b^2x^2 + a^2y^2 \]

\textbf{Odsekovna} enačba elipse v središčni legi:
\[ \frac{x^2}{a^2} + \frac{y^2}{b^2} = 1 \]
$a$ je odsek na abscisni osi oziroma \textbf{velika polos}, $b$ pa odsek na ordinatni osi
ali \textbf{mala polos}. Konstanta $e$ se imenuje \textbf{linearna ekscentričnost} in se izračuna kot:
\[ e^2 = a^2 - b^2 \]
Konstanta $\varepsilon$ se imenuje \textbf{numerična ekscentričnost}. Pri elipsi je vedno
manjša od~1.
\[ \varepsilon  = \frac{e}{a} \]
$p$ se imenuje \textbf{polparameter} elipse in je vrednost elipse pri $e$.
\[ p = \pm \frac{b^2}{a} \]

\textbf{Eksplicitna} enačba elipse:
\[ y = \pm \frac{b}{a} \sqrt{a^2 - x^2} \]
Če to enačbo primerjamo z enačbo krožnice \eqref{eq:kroz:eksp}, ugotovimo, da je elipsa
pravzaprav krožnica, raztegnjena za faktor $b/a$.

\textbf{Gorišči}:
\begin{align*}
 &G_1(-e, 0) \\
 &G_2(e, 0)
\end{align*}

\textbf{Temena} elipse:
\begin{align*}
  &A(a,0) \\
  &B(-a, 0) \\
  &C(0, b) \\
  &D(0,-b)
\end{align*}

Elipsa je \textbf{simetrična} glede na obe koordinatni osi, njeno \textbf{definicijsko območje} pa je
$D_f[-a, a]$, saj drugače $e$ ni definiran, $e^2 = a^2 - b^2 > 0$.

Vse zgornje enačbe veljajo za ``ležeče'' elipse, pri katerih je $a > b$. Če pa imamo
``pokončno'' elipso, moramo v vseh enačbah zamenjati $a$ in $b$, pa tudi gorišča so na
ordinatni osi.

Enačba elipse v \textbf{premaknjeni} legi:
\[ \frac{(x-p)^2}{a^2} + \frac{(y-q)^2}{b^2} = 1 \]
Središče elipse je v točki $S(p,q)$.

\textbf{Potreben pogoj} za elipso (do njega pridemo podobno kot pri krožnici):
\[ B = 0 \land A \krat C > 0 \comment{$A$ in $C$ sta enako predznačena} \]

\subsection{Hiperbola}
\label{sec:hiper}
\textbf{Hiperbola} je množica točk ravnine, ki imajo konstantno absolutno vrednost razlike razdalj
do dveh izbranih točk, ki ju imenujemo gorišči hiperbole.

\[ |G_1G_2| = 2e \]
\[ |r_1 - r_2| = 2a \]

Vstavi izpeljavo enačbe tukaj. Sliko tudi.

\textbf{Odsekovna} enačba hiperbole:
\[ \frac{x^2}{a^2} - \frac{y^2}{b^2} = 1 \]

$e$ se imenuje \textbf{linearna ekscentričnost} in se izračuna kot:
\[ e^2 = a^2 + b^2 \]

$\varepsilon$ se imenuje \textbf{numerična ekscentričnost} in je pri hiperboli vedno večji od 1.
\[ \varepsilon = \frac{e}{a} \]

\textbf{Eksplicitna} enačba hiperbole:
\[ y = \pm \frac{b}{a} \sqrt{x^2 - a^2} \]

\textbf{Definicijsko območje} hiperbole je $D_f = (-\infty, -a] \cup [a, \infty)$.
Hiperbola je \textbf{simetrična} glede na obe koordinatni osi.

\textbf{Gorišči} hiperbole:
\begin{align*}
 &G_1(-e, 0) \\
 &G_2(e, 0)
\end{align*}

\textbf{Temeni} hiperbole:
\begin{align*}
  &A(-a, 0) \\
  &B(a, 0)
\end{align*}

\textbf{Asimptoti} hiperbole sta premici
\[ y = \pm \frac{b}{a}x \]
\[ y = \pm \frac{b}{a} \sqrt{x^2 - a^2}  = \pm \frac{b}{a} x \sqrt{1 -
\frac{a^2}{x^2}} \]
Če gre $x$ proti $\infty$, gre vrednost ulomka $\frac{a^2}{x^2}$ proti 0 in vrednost korena v eksplicitni enačbi proti 1, iz česar
ugotovimo asimptoti.

$p$ je \textbf{polparameter} hiperbole in je vrednost parabole pri $e$.
\[ p = \pm \frac{b^2}{a} \]

\textbf{Potreben pogoj} za hiperbolo:
\[ B = 0 \land A \krat C < 0 \comment{$A$ in $C$ različno predznačena} \]

Enačba ``pokončne'' hiperbole:
\[ \frac{x^2}{a^2} - \frac{y^2}{b^2} = -1 \]

Enačba hiperbole v \textbf{premaknjeni} legi:
\[ \frac{(x-p)^2}{a^2} - \frac{(y-q)^2}{b^2} = 1 \]
Središče hiperbole je v točki $S(p,q)$.


\subsection{Parabola}
\label{sec:stoz:parab}
\textbf{Parabola} je množica točk v ravnini, ki imajo enako razdaljo od izbrane premice
\textbf{vodnice} $v$ in izbrane točke \textbf{gorišča} $G$.

\[ \mathcal{P} = \left\{ T(x,y): d(T,v) = d(T,G) \right\} \]

Vstavi skico tukaj. Izpeljavo enačbe tudi.

\textbf{Temenska} enačba parabole:
\[ y^2 = 2px \]

\textbf{Eksplicitna} enačba parabole:
\[ y = \pm \sqrt{2px} \]

$p$ je \textbf{polparameter} parabole in je enak razdalji med vodnico in goriščem.
Vrednost parabole pri $p/2$ je $p$.

\textbf{Gorišče} parabole:
\[ G\left(\frac{p}{2}, 0  \right) \]

Enačba \textbf{vodnice}:
\[ x = -\frac{p}{2} \]

\textbf{Teme} parabole:
\[ T(0, 0) \]

Parabola je \textbf{simetrična}, njena os simetrije pa se imenuje os parabole. Definicijsko območje
parabole $D_f = [0, \infty)$.

Enačba \textbf{premaknjene} parabole:
\[ (y-b)^2 = 2p(x-a) \]
Teme je v točki $T(a, b)$.

Enačba \textbf{zrcaljene}, \textbf{pokončne} in \textbf{pokončne zrcaljene} parabole v tem vrstnem redu:
\begin{align*}
  y^2 &= -2px \\
  y &= \frac{1}{2p}x^2 \\
  y &= -\frac{1}{2p}x^2
\end{align*}

\textbf{Potreben pogoj} za parabolo:
\[ A = B = 0 \lor B = C = 0 \]

\printindex
\end{document}
% vim: spell spelllang=sl
